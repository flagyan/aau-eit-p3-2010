\chapter{Appendiks B}
\label{mic_output}
\section*{Måling på Monacor MCE-4000}
Denne målerapport dokumenterer en måling på en Monacor MCE-4000 mikrofon foretaget i B3-209, Fredrik Bajers Vej 7. Målingen blev foretaget den 30. november 2010 af Benjamin Krebs og Jacob Hansen.
\subsection*{Formål}
\label{mic_output_formaal}
Målingens formål er:
\begin{itemize}
\item At bestemme peakspændingen på udgangen fra en Monacor MCE-4000 mikrofon ved 70 dB(A), i en afstand af 0,1 m.
\item At bestemme peakspændingen på udgangen fra en Monacor MCE-4000 mikrofon ved 90 dB(A), i en afstand af 0,1 m.
\end{itemize}
\subsection*{Testobjekt}
\label{mic_output_testobjekt}
Her skal testobjektet defineres entydigt. Der kan evt. henvises til diagrammer i rapporten.\\
Det er vigtigt, at alle målepunkter er veldefinerede.\\

\subsection*{Teori}
\label{mic_output_teori}
Lydtrykket for almindelig tale er 60 dB(A) i en afstand af én meter\fixme{kilde: http://www.es.aau.dk/sections/acoustics/press/fakta/lidt\_om\_lyd/}. Da det ikke forventes at brugen af en mikrofon under sang foregår ved én meters  afstand, skal denne værdi regnes om før den kan give et realistisk billede af hvilket lydtryk mikrofonen bliver udsat for. De 60 dB(A) regnes først om til Pa ved udregningen i formel (\ref{equ:db-pa1})

\begin{equation}
\label{equ:db-pa1}
p_{\mathrm{60~dB(A)}} = 10^{\frac{L_p}{20}} \cdot p_{\mathrm{ref}} = 10^{\frac{60~dB(A)}{20}} \cdot 20 \cdot 10^{-6}~\mathrm{Pa} = 0,02~\mathrm{Pa}
\end{equation}

Denne værdi er altså ved én meters afstand. Omregningen af lydtrykket til 0,1 m fra kilden foretages som i udregningen i formel (\ref{equ:distance}).

\begin{equation}
\label{equ:distance}
p_2 = p_1 \cdot \frac{r_1}{r_2} = 0,02~\mathrm{Pa} \cdot \frac{1~\mathrm{m}}{0,1~\mathrm{m}} = 0,2~\mathrm{Pa}
\end{equation}

Lydtrykket, i Pa, bliver altså, når afstanden deles med ti, ti gange større. Da dB(A) er en logaritmisk skala bliver 60 dB(A) til 80 dB(A) når afstanden reduceres fra 1 m til 0,1 m. Som forklaret i starten af kapitel \ref{forforstaerker} bliver arbejdsområdet dermed fra 70 db(A) til 90 dB(A). De forventede peakspændinger på udgangen af mikrofonen kan derfor regnes ved formel (\ref{equ:vmicpeak}), hvor lydtrykket igen er omregnet til Pa.

\begin{equation}
\label{equ:vmicpeak}
\hat{V}_{\mathrm{microphone}} = p \cdot 5\frac{\mathrm{mV}}{\mathrm{Pa}}
\end{equation}

Dette giver en teoretisk minimums og maksimumspeakspænding på udgangen på henholdsvis 3,16 mV og 31,6 mV.

\subsection*{Måleopstilling}
\label{mic_output_maaleopstilling}
Her skal vises en tegning over måleopstillingen, så man klart kan se, hvordan udstyret er tilsluttet.\\
Hvis der bruges flere forskellige tilslutninger under målingen, kan der vises flere forskellige opstillinger, eller der kan skrives en forklarende tekst.\\

\subsection*{Anvendt udstyr}
\label{mic_output_anvendtudstyr}
Alt væsentligt udstyr skal beskrives entydigt.\\

\subsection*{Måleprocedure}
\label{mic_output_maaleprocedure}
Her beskrives klart og entydigt, hvordan målingen er foretaget inkl. alle ikke indlysende indstillinger af apparater. Eks.: \\
1. Spændingsforsyningen tilsluttes og indstilles til 15 V (måles med voltmeteret) \\
2. Generatoren indstilles til at give en sinusspænding med en amplitude på 14 mV (måles med oscilloskopet) \\
3. ....\\

\subsection*{Resultater}
\label{mic_output_resultater}
Nogle resultater kan med fordel flyttes (eller kopieres) til rapporten – husk henvisning \\
Ofte angives tabeller i målejournalen og grafer i rapporten \\
Brug tabeller – resultater blandet med tekst bliver rodet\\
Datafiler bør (desuden) vedlægges rapporten på en CD – husk henvisning\\
Præcis formulering er vigtig!!\\
Angiv enheder - DC, RMS, amplitude, eller spids-spids værdier?\\

\subsection*{Måleusikkerheder}
\label{mic_output_maaleusikkerheder}
Her angives væsentlige fejlkilder og usikkerheder i.f.b. med målingen. \\
Principielt skal man medtage alle usikkerheder og lave en samlet usikkerhedsberegning, men oftest nævnes kun de mest væsentlige. \\
Det er vigtigt at forklare uoverensstemmelser mellem beregnede, simulerede og målte data, men det hører hjemme i hovedrapporten – ikke i målejournalen. \\
I rapporten kan man evt. henvise til usikkerheder beskrevet i målejournalen.\\
Typiske årsager til måleunøjagtighed:\\
Måleinstrumenter påvirker (belaster) måleobjektet\\
Aflæsningsunøjagtighed\\
Analoge (antikke) viserinstrumenter	\\
Oscilloscop-cursor (pas på støj i ”auto-peak-peak”)\\
Støj, 50 Hz (100 Hz) brum, switch-mode spændingsforsyninger m.v.\\
Instrumentets unøjagtighed: Se manualen! \\
Multimetre: Frekvensafhængig måleusikkerhed \\
Oscilloscop: Både horisontal (lille) og vertikal usikkerhed\\