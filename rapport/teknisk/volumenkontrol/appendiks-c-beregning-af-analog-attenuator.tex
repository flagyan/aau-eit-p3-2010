\chapter{Beregning af analog attenuator}
\label{beregning-af-analog-attenuator}

Den analoge attenuator, der benyttes i volumenkontrollen, består af to attenuatorer, adskilt af en buffer. Der sidder desuden en buffer efter det andet dæmpningstrin, for at garantere at trinet har en fast udgangsimpedans. Den første attenuator dæmper i syv trin med 8 dB mellem hvert trin, startende fra 0 dB. Der tages i beregningerne udgangspunkt i formlen for en ubelastet spændingsdeler, hvor forholdet mellem ind- og udgangsspænding er noteret som dæmpningen i dB omregnet til antal gange. Beregningerne for den første attenuatorer er opstillet i ligningerne (\ref{beregning-af-analog-attenuator-ligning1}) til (\ref{beregning-af-analog-attenuator-ligning5}).

\begin{equation}
\label{beregning-af-analog-attenuator-ligning1}
\frac{R_7}{R_1 + R_2 + R_3 + R_4 + R_5 + R_6 + R_7} = 10^{\frac{-48}{20}}
\end{equation}

\begin{equation}
\frac{R_7 + R_6}{R_1 + R_2 + R_3 + R_4 + R_5 + R_6 + R_7} = 10^{\frac{-40}{20}}
\end{equation}

\begin{equation}
\frac{R_7 + R_6 + R_5}{R_1 + R_2 + R_3 + R_4 + R_5 + R_6 + R_7} = 10^{\frac{-32}{20}}
\end{equation}

\begin{equation}
\frac{R_7 + R_6 + R_5 + R_4}{R_1 + R_2 + R_3 + R_4 + R_5 + R_6 + R_7} = 10^{\frac{-24}{20}}
\end{equation}

\begin{equation}
\frac{R_7 + R_6 + R_5 + R_4 + R_3}{R_1 + R_2 + R_3 + R_4 + R_5 + R_6 + R_7} = 10^{\frac{-16}{20}}
\end{equation}

\begin{equation}
\label{beregning-af-analog-attenuator-ligning5}
\frac{R_7 + R_6 + R_5 + R_4 + R_3 + R_2}{R_1 + R_2 + R_3 + R_4 + R_5 + R_6 + R_7} = 10^{\frac{-8}{20}}
\end{equation}

Da der er seks ligninger med syv ubekendte bestemmes $R_7$ til 1 k\ohm, det resulterer i følgende værdier 
$R_1 = 162 ~\mathrm{k}\ohm, R_2 = 48,8 ~\mathrm{k}\ohm, R_3 = 24,0 ~\mathrm{k}\ohm, R_4 = 9,54 ~\mathrm{k}\ohm, R_5 = 3,80  ~\mathrm{k}\ohm, R_6 = 1,51 ~\mathrm{k}\ohm$ og $R_7 = 1,00 ~\mathrm{k}\ohm$.

Den anden attenuator dæmper i otte trin med 1 dB mellem hvert trin, startende fra 0 dB. Beregningerne for den første attenuatorer er opstillet i ligningerne (\ref{beregning-af-analog-attenuator-ligning6}) til (\ref{beregning-af-analog-attenuator-ligning12}).

\begin{equation}
\label{beregning-af-analog-attenuator-ligning6}
\frac{R_{14}}{R_{14} + R_{13} + R_{12} + R_{11} + R_{10} + R_{9} + R_{8} + R_{7}} = 10^{\frac{-7}{20}}
\end{equation}

\begin{equation}
\frac{R_{14} + R_{13}}{R_{14} + R_{13} + R_{12} + R_{11} + R_{10} + R_{9} + R_{8} + R_{7}} = 10^{\frac{-6}{20}}
\end{equation}

\begin{equation}
\frac{R_{14} + R_{13} + R_{12}}{R_{14} + R_{13} + R_{12} + R_{11} + R_{10} + R_{9} + R_{8} + R_{7}} = 10^{\frac{-5}{20}}
\end{equation}

\begin{equation}
\frac{R_{14} + R_{13} + R_{12} + R_{11}}{R_{14} + R_{13} + R_{12} + R_{11} + R_{10} + R_{9} + R_{8} + R_{7}} = 10^{\frac{-4}{20}}
\end{equation}

\begin{equation}
\frac{R_{14} + R_{13} + R_{12} + R_{11} + R_{10}}{R_{14} + R_{13} + R_{12} + R_{11} + R_{10} + R_{9} + R_{8} + R_{7}} = 10^{\frac{-3}{20}}
\end{equation}

\begin{equation}
\frac{R_{14} + R_{13} + R_{12} + R_{11} + R_{10} + R_{9}}{R_{14} + R_{13} + R_{12} + R_{11} + R_{10} + R_{9} + R_{8} + R_{7}} = 10^{\frac{-2}{20}}
\end{equation}

\begin{equation}
\label{beregning-af-analog-attenuator-ligning12}
\frac{R_{14} + R_{13} + R_{12} + R_{11} + R_{10} + R_{9} + R_{8}}{R_{14} + R_{13} + R_{12} + R_{11} + R_{10} + R_{9} + R_{8} + R_{7}} = 10^{\frac{-1}{20}}
\end{equation}

Da der er syv ligninger med otte ubekendte bestemmes $R_{14}$ til 10 k\ohm, hvilket resulterer i følgende værdier 
$R_{8} = 2,43 ~\mathrm{k}\ohm, R_{9} = 2,16 ~\mathrm{k}\ohm, R_{10} = 1,93 ~\mathrm{k}\ohm, R_{11} = 1,72 ~\mathrm{k}\ohm, R_{12} = 1,53 ~\mathrm{k}\ohm, R_{13} = 1,36 ~\mathrm{k}\ohm, R_{14} = 1,22 ~\mathrm{k}\ohm$ og $R_{15} = 10,0 ~\mathrm{k}\ohm$.