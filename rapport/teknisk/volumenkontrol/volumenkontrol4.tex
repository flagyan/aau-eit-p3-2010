\section{Implementering}
Under implementering af volumen kontrollen, kunne der konkluderes at der var en række problemer med designet. Det kunne konkluderes at spændingen på udgangen af at LM324 ikke blev trukket helt lav, efter den sidste AND-gate blev monteret. Dette betød at transistoren åbnede konstant, hvilket umuliggjorde svingning. Efter at have undersøgt databladet for LM324, blevet det konstateret at den kun kan tage 50 $\mu$A, ind i indgangen, ved en meget lav spænding. Derfor blev det besluttet at skifte $R_{16}$ på $48,5~k\ohm$ ud med en modstand på $487~k\ohm$ , for at sænke strømmen der løb med en faktor 10, hvilket både vil give et lavere spændingsfald over dioden samt sørge for det løber en mindre strøm ind i opampen. Dette løste problemet. Der blev også diskuteret at benytte en buffer i stedet, men dette blev nedstemt, da dette ville betyde en større ændring, end at skifte en modstand.

Det kunne yderligere konstateres, at der findes en bug i designet, der dog er yderst sjælden. Hvis man slipper en knap mens spændingen på udgangen af integratoren er nedadgående, vil Schmidt-triggeren være lav, hvilket betyder at transistoren vil være lukket. Da signalet fra schmidttriggeren AND'es sammen med signalet fra XOR-gaten, vil det er ikke være muligt at få et clock output til tælleren. Der kan derfor argumenteres for en høj dutycycle, for at formindske chancen for at slippe knappen mens schmidt-triggeren er lav. Da tælleren alligevel er triggered på opadgående flanker, vil der kun komme ét output signal pr. periode. For at komme ud af denne bug, kan en knap holdes nede indtil ladekondensatoren kommer over diodespændingen så der igen kommer en input spænding på integratoren, hvilket vil starte svingningen igen. 
For at modvirke det, sættes en modstand, $R_{\mathrm{stor}}$, fra minusbenet på integratoren til forsyningsspændingen. Dette vil betyde at kondensatoren kan oplade, hvilket betyder at minus-indgangen på schmidt-triggeren vil gå lav. Når minus indgangen er lavere end plus indgangen, vil udgangen gå høj, hvilket vil åbne transistoren. Når transistoren vil der være en spændingsdeling imellem $R_{\mathrm{stor}}$ og $R_4$ parallelt med $R_1$, $R_2$ og $R_3$ i serie. Der vil ligge en meget lille spænding på minus indgangen, da størstedelen af spændingen vil ligge over den største modstand. Det vigtige er dog, at spændingen på plus-indgangen er endnu mindre. Dette vil tvinge udgangen på op-ampen nedad. Denne opstilling vil sørge for at outputtet på integratoren er lavt, hvilket vil sørge for at schmidttriggeren vil være høj. Hvis schmidttriggerens output ikke havde styret transistoren ville integratoren stadig være outputte lavt, da der så bare vil sidde en spændingsdeler bestående af $R_{\mathrm{stor}}$, $R_1$, $R_2$ og $R_3$.
For at denne store modstand $R_{\mathrm{stor}}$ ikke skal have noget at sige for svingningsforløbet er den valgt til at være 100 gange større end $R_1$. 