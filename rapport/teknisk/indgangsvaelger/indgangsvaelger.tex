\chapter{Indgangsvælger}
\label{indgangsvaelger}
Indgangsvælgerens opgave er at gøre brugeren i stand til at vælge hvilke af indgangssignalerne der ønskes afspillet. Den skal altså med andre ord være i stand til at slukke for det eller de signaler brugeren ikke ønsker at høre og lukke de andre signaler igennem. \\
En måde dette kunne gøres på er at benytte en multiplexer til at vælge imellem de forskellige signaler. Det er dog blevet besluttet at der så vidt muligt skal benyttes diskrete komponenter frem for integrerede kredse, hvormed multiplexeren blev fravalgt.
Der er i tabel \ref{tab:krav_indgangsvaelger} opstillet de relevante krav til indgangsvælgeren fra kravspecifikationen. 

\begin{table}[h]
\centering
\begin{tabular}{l|r}
\hline\hline
Område & Krav \\
\hline\hline
Antal trin i & 4 \\
indgangsvælgeren & \\[4pt]
Indgangsimpedans & > 22 k\ohm \\[4pt]
Frekvensgang & $\pm$ 0,375 dB ved 20 Hz - 20 kHz, ref. 1 kHz \\
& $\pm$ 0,75 dB fra 20 Hz til 63 Hz \\
& $\pm$ 0,75 dB fra 12,5 kHz til 20 kHz \\[4pt]
Dæmpning af slukket & > 50 dB ved 1 kHz \\
indgangssignal & \\
\hline\hline
\end{tabular}
\caption{Krav til indgangsvælgeren}
\label{tab:krav_indgangsvaelger}
\end{table}



