\section{Beregning af forforstærker}
Forforstærkeren opbygges af to common-emitter trin med uafkoblet emittermodstand. Forstærkningen for hele forforstærkeren er bestemt af mikrofonens maksimale spændingssving samt hvad signal skal forstærkes op til; i dette tilfælde linieniveau. Linieniveau har, ifølge standarden \fixme{kilde til den korrekte standard?}, et spændingssving fra 200 mV til 2 V. Mikrofonen som benyttes, MCE-4000, har en følsomhed på 5 mV/Pa ved 1kHz. En meget kraftig lyd har et lydtryk på 20 Pa \fixme{kilde til http://www.es.aau.dk/sections/acoustics/research/projects/gener_fra_lavfrekvent_stoej/fakta_om_lyd/} hvormed den maksimale outputspænding fra mikrofonen bliver, som vist på ligning \ref{eq:mikrofonoutput}.

\begin{equation}
\label{eq:mikrofonoutput}
20 Pa \cdot 5 \frac{mV}{Pa} = 100 mV_{Peak}
\end{equation}

\subsection*{Kredsløb til mikrofon}

Mikrofonen som benyttes behøver, i følge databladet, en supply spænding fra 1,5 til 10 V og en strøm på 0,5 mA. Desuden skal koblingskondensatorens størrelse være mellem 0,1-4,7$\mu$F



\subsection*{Trin 2)

Den maksimale collectormodstand ($R_c$) kan beregnes ud fra at få så høj råforstærkning som muligt. Grunden til at der stræbes efter stor råforstærkning er at have så meget som muligt at kunne tilbagekoble for derved at sænke THD. Den maksimale $R_c$ er givet ved ligning \ref{eq:rcmaks}.


\begin{equation}
\label{eq:rcmaks}

R_c = \sqrt{\frac{ V_{R_c} \cdot R_sig \cdot R_L }{ h_{FE} \cdot V_t}}

\end{equation}