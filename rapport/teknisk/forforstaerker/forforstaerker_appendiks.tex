\chapter{Appendiks B}
\section*{Bestemmelse af Monacor MCE-4000 output}
Lydtrykket for almindelig tale er 60 dB(A) i en afstand af én meter\fixme{kilde: http://www.es.aau.dk/sections/acoustics/press/fakta/lidt\_om\_lyd/}. Da det ikke forventes at brugen af en mikrofon under sang foregår ved én meters  afstand, skal denne værdi regnes om før den kan give et realistisk billede af hvilket lydtryk mikrofonen bliver udsat for. De 60 dB(A) regnes først om til Pa ved udregningen i formel (\ref{equ:db-pa1})

\begin{equation}
\label{equ:db-pa1}
p_{\mathrm{60~dB(A)}} = 10^{\frac{L_p}{20}} \cdot p_{\mathrm{ref}} = 10^{\frac{60~dB(A)}{20}} \cdot 20 \cdot 10^{-6}~\mathrm{Pa} = 0,02~\mathrm{Pa}
\end{equation}

Denne værdi er altså ved én meters afstand. Omregningen af lydtrykket til 0,1 m fra kilden foretages som i udregningen i formel (\ref{equ:distance}).

\begin{equation}
\label{equ:distance}
p_2 = p_1 \cdot \frac{r_1}{r_2} = 0,02~\mathrm{Pa} \cdot \frac{1~\mathrm{m}}{0,1~\mathrm{m}} = 0,2~\mathrm{Pa}
\end{equation}

Lydtrykket, i Pa, bliver altså, når afstanden deles med ti, ti gange større. Da dB(A) er en logaritmisk skala bliver 60 dB(A) til 80 dB(A) når afstanden reduceres fra 1 m til 0,1 m. Som forklaret i starten af kapitel \ref{forforstaerker} bliver arbejdsområdet dermed fra 70 db(A) til 90 dB(A). De forventede peakspændinger på udgangen af mikrofonen kan derfor regnes ved formel (\ref{equ:vmicpeak}), hvor lydtrykket igen er omregnet til Pa.

\begin{equation}
\label{equ:vmicpeak}
\hat{\mathrm{V}}_{\mathrm{microphone}} = p \cdot 5\frac{\mathrm{mV}}{\mathrm{Pa}}
\end{equation}

Dette giver en teoretisk minimums og maksimumspeakspænding på udgangen på henholdsvis 3,16 mV og 31,6 mV.