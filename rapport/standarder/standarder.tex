\section{Standarder}
Formålet med dette afsnit er at beskrive forskellige standarder der er relevante for opbygningen af en HiFi-forstærker, for dernæst at det kan udmunde i en kravspecifikation på baggrund af de valgte standarder . Standarder gruppen har valgt at arbejde med er :

\begin{itemize}              
\item IEC61938-1             
\item DIN 45500 normen    
\end{itemize} 

Der er taget udgangspunkt i IEC61938-1. IEC61938-1 er sidst opdateret i 1-12-1997 \ref{??}\fixme{ref to standard IEC 61938-1}. DIN 45500 normen er taget med i projektet selvom den er forældet, sidst opdateret i 01-1973 \ref{??}\fixme{ref to standard  DIN 45500 normen}. Det er gjort fordi der ofte refereres til den i de nyere standarder. Disse standarder er blevet undersøgt for forskellige parametre, hvorefter de nyeste eller mest relevante data er blevet valgt ud til at arbejde videre med.
\newline
\newline
Tabel \ref{tab:standarder_krav} angiver de parametre standarderne er undersøgt for, samt hvilken specifik standard der er valgt til hver parameter.

\begin{table}[h]
\centering
\begin{tabular}{l|l|l}
\hline\hline
Område & Minimumskrav & Standard \\
\hline\hline
Udgangseffekt & Min. 10 W mono $^{[1]}$ & DIN45500 \\
& Min. 2x6 W stereo $^{[1]}$ & \\
\hline
Frekvens omrade & 40 Hz - 16000Hz & DIN45500 \\
\hline
THD & Max 1 \% $^{[2]}$ & DIN45500 \\
& Max 0.7 \% $^{[3]}$ & \\
\hline
Belastningsimpedans & & DIN45500 \\
Højtaler & 4 eller 8 \ohm~$^{[4]}$ & \\
Hovedtelefon & 200 eller 400 \ohm~$^{[4]}$ & \\
\hline
Indgangssignal & 0.5 V & IEC 61938-1 \\
\hline
Indgangsimpedans & Min. 22 k\ohm & IEC61938-1 \\
\hline
Udgangsimpedans & Max. 2.2 k\ohm & IEC61938-1 \\
\hline
Udgangsspænding & Max. 2 V & IEC61938-1 \\
& Min. 0.2 V & IEC61938-1 \\
\hline\hline
\end{tabular}
\caption{Tabel over minimumskrav fra standarder.}
\label{tab:standarder_krav}
\end{table}

\begin{itemize}
\item[]{[1] 1000 Hz sinus signal i 10 min. Ved en omgivelse temperatur på 35°C}
\item[]{[2] Forforstærker og effektforstærker. Målt i effektbåndbredde 40 - 12500 Hz, med en udgangseffekt på minimum 10 W}
\item[]{[3] Forforstærker eller effektforstærker. Målt i effektbåndbredde 40 - 12500 Hz, med en udgangseffekt på minimum 10 W}
\item[]{[4]Tolerance på 20 \%}
\end{itemize}

Der er i dette afsnit blevet opstillet en række krav fra de undersøgte standarder. Disse data vil blive brugt videre i kravspecifikationen.