\chapter{Kravspecifikation}
\label{kravspec}
Formålet med dette kapitel er at opstille en kravspecifikation for projektets HiFi-forstærker. Alle kravene i kravspecifikationen skal være målbare, så de kan testes ved projektets afslutning, og begrundede i det omfang dette er muligt. 

Tabel \ref{tab:kravspec} viser hvilke krav der er stillet til dette projekts HiFi-forstærker. Tabellen viser desuden videre til hvilke begrundelser, der er for hvert enkelt krav.
\begin{table}[h]
\centering
\begin{tabular}{l|r|l}
\hline\hline
Område & Krav & Baggrund for krav \\
\hline\hline
\textbf{Teknisk:} & & \\
Forstærkerklasse & AB & Se ref til klasse afsnit \\
Total Harmonisk Forvrænging & <1 \% & Se ref til THD afsnit \\
Indgange & Linie og mikrofon & Se \ref{indgange} \\
Indgangsvælger & 3 trin \tiny{(On/Off, Off/On, On/On)} & Se \ref{krav_indgangsvaelger} \\
Indgangsimpedans & 47,5 k\ohm \tiny{($\pm$ 1 \%)}& Se \ref{krav_indgangsimpedans} \\
Equalizer-niveauer & ? & ? \\
Volumenkontrol & Digital \tiny{(XX niveauer)} & Se \ref{krav_volumenkontrol} \\
Udgangseffekt & 20 Watt \tiny{($\pm$ 2 Watt)} \normalsize{i 8$\Omega$} & Se \ref{krav_udgangseffekt} \\
Udgangssignaltype & Mono & Se \ref{krav_udgangssignal} \\
Udgangsimpedans & ? & Standard eller klasse \\
Kortslutningssikring & ? & ? \\
\hline
\textbf{Brugerflade:} & & \\
Volumedisplay & Ja & Se \ref{krav_volumenkontrol} \\
Indgangsvælger på forstærkeren & Ja & Se \ref{krav_indgangsvaelger} \\
Visualizer & Ja & Se ref til Visualizer underafsnit \\
\hline
\textbf{Fjernbetjening:} & & \\
Volumenkontrol & Ja & \ref{krav_fjernbetjening} \\
Indgangsvælger & Ja & \ref{krav_fjernbetjening} \\
Rækkevidde & ? & ? \\
\hline\hline
\end{tabular}
\caption{Tabel over samlet kravspecifikation}
\label{tab:kravspec}
\end{table}

\section{Udgangseffekt}
\label{krav_udgangseffekt}
Fastsættelsen af udgangseffektens størrelse er bestemt af to faktorer. Den maksimale effekt der er mulig er bestemt af sikkerhedsreglerne i elektronikværkstedet på Aalborg Universitet. I disse regler angives den maksimale DC spænding der må arbejdes med til 60 V. Til projektets forstærker deles denne spænding til en $\pm$30 V forsyning. Under udregningen af den maksimale effekt bruges RMS-værdien (Root Mean Square) af den spænding. Dermed bliver den øvre grænse som vist i udregning \ref{equ:maks_effekt}.

\begin{equation}
\label{equ:maks_effekt}
P_{maks} = \frac{V^2}{R}= \frac{(\frac{\hat{V}}{\sqrt{2}})^2}{R} = \frac{(\frac{30~V}{\sqrt{2}})^2}{8\ohm} = 56,25~W
\end{equation}

Den nedre grænse for udgangseffekten er defineret af standarden DIN45500, i hvilken det er bestemt at udgangseffekten som minimum skal være 10 Watt i mono før forstærkeren må kaldes en HiFi-forstærker.

Tolerancen er valgt til 10 \%, da dette er en tolerancestørrelse man tit ser i elektroniske komponenter.

\section{Indgangsvælger}
\label{krav_indgangsvaelger}
Skal vel begrundes ud fra et læringsmæssigt synspunkt

\section{Indgangsimpedans}
\label{krav_indgangsimpedans}
Indgangsimpedansen er at opfatte som en impedans der, ud fra en almindelig spændingsdeling, reducerer\fixme{Hvad kalder man det?} indgangssignalet. Man er, med den begrundelse, interesseret i et stor indgangsimpedans. Den mindste tilladte indgangsimpedans for en HiFi-forstærker er i standard IEC61938-1 bestemt til 22 k\ohm. Da udgangsimpedansen samtidig er bestemt til 2,2 k\ohm~ for et liniesignal, kan betydningen af indgangsimpedansens størrelse regnes som vist i udregning \ref{equ:indgangsimpedans22}. 

\begin{equation}
\label{equ:indgangsimpedans22}
\frac{22~k\ohm}{22~k\ohm + 2,2~k\ohm} = 0,91
\end{equation}

Det ses af udregning \ref{equ:indgangsimpedans22} at en indgangsimpedans på 22 k\ohm~ vil medføre et indgangssignal på 91 \% af det oprindelige signal. Med en større indgangsimpedans vil en større del af det oprindelige signal blive indgangssignalet. 

\begin{equation}
\label{equ:indgangsimpedans475}
\frac{47,5~k\ohm}{47,5~k\ohm + 2,2~k\ohm} = 0,96
\end{equation}

Udregning \ref{equ:indgangsimpedans475} viser at med en indgangsimpedans på 47,5 k\ohm~ bliver indgangssignalet hele 96 \% af det oprindelige signal.

\section{Udgangssignal}
\label{krav_udgangssignal}
Argumentationen skal gøre på at stereo bare vil fordoble arbejdet og ikke gøre noget for indlæringen

\section{Volumenkontrol}
\label{krav_volumenkontrol}
Begrundes med et minimum mennesket er i stand til at skelne i forhold til hvor stor et område vi skal $"$dække$"$

\section{Fjernbetjening}
\label{krav_fjernbetjening}
