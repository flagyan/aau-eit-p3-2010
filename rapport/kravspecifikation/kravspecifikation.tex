\chapter{Kravspecifikation}
\label{kravspec}
Formålet med dette kapitel er at opstille en kravspecifikation for projektets HiFi-forstærker. Alle kravene i kravspecifikationen skal være målbare, så de kan testes ved projektets afslutning, og begrundede i det omfang dette er muligt. 

Tabel \ref{tab:kravspec} viser hvilke krav der er stillet til dette projekts HiFi-forstærker. Tabellen viser desuden videre til hvilke begrundelser der er for hvert enkelt krav.
\begin{table}[h]
\centering
\begin{tabular}{l|r|l}
\hline\hline
Område & Krav & Baggrund for krav \\
\hline\hline
\textbf{Teknisk:} & & \\
Forstærkerklasse & AB & Se ref til klasse afsnit \\
Total Harmonisk Forvrænging & <1\% & Se \ref{krav_forvraengning} \\
Indgange & Line og Mikrofon & Se \ref{indgange} \\
Indgangsvælger & Flydende/Trin ? & Se \ref{krav_indgangsvaelger} \\
Indgangsimpedans & ? & Standard eller klasse \\
Equalizer-niveauer & ? & ? \\
Volumenkontrol & Digital (XX niveauer) & Se \ref{krav_volumenkontrol} \\
Udgangseffekt & 20 Watt ($\pm$ 2 Watt) i 8$\Omega$ & Se \ref{krav_udgangseffekt} \\
Udgangssignal \fixme{Hvad hedder det?} & Mono & Se \ref{krav_udgangssignal} \\
Udgangsimpedans & ? & Standard eller klasse \\
Kortslutningssikring & ? & ? \\
\hline
\textbf{Brugerflade:} & & \\
Volumedisplay & Ja & Se \ref{krav_volumenkontrol} \\
Indgangsvælger på forstærkeren & Ja & Se \ref{krav_indgangsvaelger} \\
Visualizer & Ja & ? \\
\hline
\textbf{Fjernbetjening:} & & \\
Volumenkontrol & Ja & \ref{krav_fjernbetjening} \\
Indgangsvælger & Ja & \ref{krav_fjernbetjening} \\
Rækkevidde & ? & ? \\
\hline\hline
\end{tabular}
\caption{Tabel over samlet kravspecifikation}
\label{tab:kravspec}
\end{table}

\section{Udgangseffekt}
\label{krav_udgangseffekt}
Fastsættelsen af udgangseffektens størrelse er bestemt af to faktorer. Den maksimale effekt der er mulig er bestemt af sikkerhedsreglerne i elektronikværkstedet på Aalborg Universitet.\fixme{Kilde: elregler_b1101.doc} I disse regler angives den maksimale AC spænding der må arbejdes med til 25 V. Dermed bliver den øvre grænse som vist i udregning \ref{equ:maks_effekt}.

\begin{equation}
\label{equ:maks_effekt}
P_{maks} = \frac{Noget}{Andet} = XX~W
\end{equation}

Den nedre grænse for udgangseffekten er bestemt af \textbf{\textit{STANDARD?!?}}, i hvilken det er bestemt at udgangseffekten som minimum skal være 10 Watt før forstærkeren må kaldes en HiFi-forstærker.

Tolerancen er valgt til 10\%, da dette er en tolerancestørrelse man tit ser i elektroniske komponenter.

\section{Forvrængning}
\label{krav_forvraengning}
Klasse afhængig? Min. grænse øret kan høre? Maks. 1\% før det er HiFi.

\section{Indgangsvælger}
\label{krav_indgangsvaelger}
Skal vel begrundes ud fra et læringsmæssigt synspunkt

\section{Udgangssignal}
\label{krav_udgangssignal}
Argumentationen skal gøre på at stereo bare vil fordoble arbejdet og ikke gøre noget for indlæringen

\section{Volumenkontrol}
\label{krav_volumenkontrol}
Begrundes med et minimum mennesket er i stand til at skelne i forhold til hvor stor et område vi skal $"$dække$"$

\section{Fjernbetjening}
\label{krav_fjernbetjening}
