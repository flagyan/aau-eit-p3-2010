\chapter{Valg af løsning}
\label{valgafloesning}
Formålet med dette kapitel er til slut at opstille en kravspecifikation for projektets HiFi-forstærker. Alle kravene i kravspecifikationen skal være målbare, så de kan testes ved projektets afslutning, og begrundede i det omfang dette er muligt. Før det er muligt at opstille en sådan kravspecifikation, er det nødvendigt at dokumentere hvilke overvejelser som danner grundlag for de forskellige dele af kravspecifikationen. Der er i kapitel \ref{indledende_analyse} ført dokumentation for en række af disse overvejelser. De resterende overvejelser er dokumenteret i de næste seks afsnit. I afsnit \ref{kravspecifikation} er produktet af alle disse overvejelser samlet i den endelige kravspecifikation for projektets HiFi-forstærker. 

\section{Indgangsvælger}
\label{valg_indgangsvaelger}
I forbindelse med indgangsvælgeren er overvejelserne gået på, hvorvidt denne skal lave en trinvis eller flydende overgang mellem indgangssignalerne. Da en flydende overgang i princippet er simultan volumenkontrol af indgangene, adskiller den form for indgangsvælger sig ikke i samme grad fra en egentlig volumenkontrol, som det er tilfældet med en trinvis indgangsvælger. Eftersom der er opstillet krav om en volumenkontrol til forstærkeren sættes kravet om indgangsvælgerens art til trinvis. \\
Som det fremgår i afsnit \ref{systemopbygning} skal HiFi-forstærkeren have to indgange, mikrofon og linie, hvilket danner grundlag for at kravet til antallet af trin i indgangsvælgeren sættes til fire. De fire trin er vist i tabel \ref{tab:indgangsvaelgertrin}.\fixme{Jesper: Har ikke lavet det med isolation, da jeg vil vente og se hvad Jacob skriver om det - ¿ Mener jeg hørte han fandt noget med det til standarder ?}

\begin{table}[h]
\centering
\begin{tabular}{c|c|c}
\hline\hline
Trin & Indgang 1 & Indgang 2 \\
\hline\hline
1 & On & On \\
2 & On & Off \\
3 & Off & On \\
4 & Off & Off \\
\hline\hline
\end{tabular}
\caption{Indgangsvælgertrin}
\label{tab:indgangsvaelgertrin}
\end{table}

Valget mellem de fire trin skal kunne tages af brugeren på HiFi-forstærkerens frontpanel. Det skal desuden være tydeligt hvilket trin indgangsvælgeren er sat på.

\section{Indgangsimpedans}
\label{valg_indgangsimpedans}
Indgangsimpedansen er at opfatte som en impedans der, ud fra en almindelig spændingsdeling, reducerer indgangssignalet. Man er, med den begrundelse, interesseret i en stor indgangsimpedans. Den mindste tilladte størrelse af indgangsimpedansen for en HiFi-forstærker er i standard IEC61938-1 bestemt til 22 k\ohm~ for liniesignalsindgange, se afsnit \ref{standarder}. Da størrelsen af udgangsimpedansen samtidig er bestemt til maksimalt 2,2 k\ohm~ for en liniesignalsudgang, kan betydningen af indgangsimpedansens størrelse regnes som vist i udregningen i formel (\ref{equ:indgangsimpedans22}). 

\begin{equation}
\label{equ:indgangsimpedans22}
\frac{22~k\ohm}{22~k\ohm + 2,2~k\ohm} = 0,91
\end{equation}

Det ses af udregningen i formel (\ref{equ:indgangsimpedans22}), at en indgangsimpedans af størrelsen 22 k\ohm~ vil medføre et indgangssignal på 91 \% af det oprindelige signal. Med en større indgangsimpedans vil en større del af det oprindelige signal blive indgangssignalet. 

\section{Volumenkontrol}
\label{valg_volumenkontrol}
Kravet til styringen af volumenkontrol er sat til at dette skal foregå digitalt. Begrundelsen herfor ligger i projektets undertema, $"$High Fidelity (Hi-Fi) forstærker med digital styring$"$, og begrundes derfor ikke yderligere. Volumen skal kunne justeres via HiFi-forstærkerens frontpanel, hvor det også skal være muligt at aflæse det øjeblikkelige volumenniveau.  

\section{Udgangssignaltype}
\label{valg_udgangssignaltype}
Valget står for udgangssignaltypen mellem stereo og mono. Da stereo i princippet blot er et lydsignal med to kanaler i modsætning til mono, som er én kanal, vil fremstillingen af en stereoudgang på forstærkeren ikke umiddelbart være mere lærerig end fremstillingen af en monoudgang, den vil blot kræve mere tid. Af den årsag vælges udgangssignaltypen til mono.

\section{Udgangseffekt}
\label{valg_udgangseffekt}
Fastsættelsen af udgangseffektens størrelse er bestemt af to faktorer. Den maksimale effekt der er mulig er bestemt af sikkerhedsreglerne i elektroniklaboratoriet på Aalborg Universitet. I disse regler angives den maksimale DC spænding der må arbejdes med til 60 V \cite{elregler-b1101}. 
Til projektets forstærker deles denne spænding til en $\pm$30 V forsyning. Under udregningen af den maksimale effekt bruges RMS-værdien (Root Mean Square) af den spænding. Desuden anvendes den, i afsnit \ref{standarder} valgte, belastningsmodstand på 8~\ohm. Dermed bliver den øvre grænse som vist i udregningen i formel (\ref{equ:maks_effekt}).

\begin{equation}
\label{equ:maks_effekt}
P_{max} = \frac{(V_{RMS})^2}{R_{load}}= \frac{(\frac{\hat{V}}{\sqrt{2}})^2}{R_{load}} = \frac{(\frac{30~V}{\sqrt{2}})^2}{8~\ohm} = 56,25~W
\end{equation}

Den nedre grænse for udgangseffekten er defineret af standarden IEC581, i hvilken det er bestemt at udgangseffekten som minimum skal være 10 W, hvis der er tale om en monoudgang, før forstærkeren må kaldes en HiFi-forstærker, se afsnit \ref{standarder}. For ikke at kommer under den nedre grænse er det valgt at udgangseffekten skal være 20 W

\section{Kortslutningssikring}
\label{valg_kortslutningssikring}
Der er opstillet krav om en kortslutningssikring for at sikre mod skader ved eventuelle overbelastninger på udgangen af HiFi-forstærkeren. Sikringen skal slå fra hvis strømmen i belastningen bliver for stor. For stor er i dette tilfælde givet ved sammenhængen mellem spændingsforsyningen og belastningsmodstanden, som udtryk i udregningen i formel (\ref{equ:sikringsstroem}).

\begin{equation}
\label{equ:sikringsstroem}
I_{max}=\frac{V_{CC}}{R_{load}}=\frac{30~V}{8~\ohm}=3,75~A
\end{equation}

\section{Kravspecifikation}
\label{kravspecifikation}
Tabel \ref{tab:kravspec} viser hvilke krav der er stillet til dette projekts HiFi-forstærker. Tabellen viser desuden videre til hvilke overvejelser eller standarder, der danner grundlag for hvert enkelt krav.

\begin{table}[h]
\centering
\begin{tabular}{r|l|r|l}
\hline\hline
Nr. & Område & Krav & Betingelse(r) \\
\hline\hline
\multicolumn{4}{c}{\textbf{Teknisk:}} \\\hline
1 & Forstærkerklasse & AB & \\[4pt]
2 & Total Harmonic & < 1 \% & $\circ$ < 0,5\% i forforstærker \\
& Distortion & & $\circ$ < 0,5 \% i effektforstærker \\
& & & $\circ$ < 0,7 \% i forforstærker \\
& & & ~~~effektforstærker \\[4pt]
3 & Frekvensgang & 20 Hz - 20 kHz & $\circ$ Maks. dæmpning på 3 dB \\
& & & ~~~fra 63 Hz til 20 Hz og \\
& & & ~~~fra 12,5 kHz til 20 kHz \\
& & & $\circ$ $\pm$ 1,5 dB før equalizer \\
& & & $\circ$ $\pm$ 2 dB efter equalizer \\[4pt]
4 & Indgangstyper & Linie og mikrofon & $\circ$ Med $"$MCE-4000 - Elec- \\
& & & ~~~tret Mike Capsul$"$ mikrofon \\[4pt]
5 & Antal trin i & 4 & \\
& indgangsvælger & & \\[4pt]
6 & Dæmpning af slukket & > 50 dB & $\circ$ Ved 1 kHz \\
& indgangssignal & & \\[4pt]
7 & Indgangsimpedans & > 22 k\ohm ($\pm$ 1 \%) & \\[4pt]
8 & Equalizer-bånd & 3 & $\circ$ Ved bånd: 20 - 200 Hz, \\
& & & ~~~200 - 2000 Hz og \\
& & & ~~~2000 - 20000 Hz \\[4pt]
9 & Volumenkontrol & Digital & \\[4pt]
10 & Udgangseffekt & 20 W ($\pm$ 2 W) & $\circ$ I 8~\ohm-højtaler \\[4pt]
11 & Udgangssignaltype & Mono & \\[4pt]
12 & Kortslutningssikring & 3,75 A & $\circ$ Med 8~\ohm-højtaler \\\hline
\multicolumn{4}{c}{\textbf{Frontpanel (input):}} \\\hline
13 & Indgangsvælger & Trykknap & \\[4pt]
14 & Volumenkontrol & 2 trykknapper & \\[4pt]
15 & Equalizer & 1 drejeknap pr. bånd & \\\hline
\multicolumn{4}{c}{\textbf{Frontpanel (output):}} \\\hline
16 & Volumedisplay & 7-segment & \\[4pt]
17 & Visualiser & Lysdioder \\
\hline\hline
\end{tabular}
\caption{Samlet kravspecifikation}
\label{tab:kravspec}
\end{table}

Med denne kravspecifikation er der nu grundlag for at udvikle og fremstille en HiFi-forstærker.