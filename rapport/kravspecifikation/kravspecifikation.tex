\chapter{Kravspecifikation}
\label{kravspec}
Formålet med dette kapitel er at opstille en kravspecifikation for projektets HiFi-forstærker. Alle kravene i kravspecifikationen skal være målbare, så de kan testes ved projektets afslutning, og begrundede i det omfang dette er muligt. 

Tabel \ref{tab:kravspec} viser hvilke krav der er stillet til dette projekts HiFi-forstærker. Tabellen viser desuden videre til hvilke begrundelser der er for hvert enkelt krav.
\begin{table}[h]
\centering
\begin{tabular}{l|r|l}
\hline\hline
 & Krav & Baggrund for krav \\
\hline\hline
Blabla & 42 & FORDI! \\
Squash & 7 & Why not? \\
\hline
Apples & Yes & \\
\hline\hline
\end{tabular}
\caption{Tabel over samlet kravspecifikation}
\label{tab:kravspec}
\end{table}
\section*{Effektforstærker}
\begin{enumerate}
\item Udgangseffekten skal være 20 $\pm$ 2 Watt i 8$\Omega$
\item Forvrængning? (Wiki: Forvrængning kan ikke høres under 1\%)
\item Niveauer på equaliser?
\item Visualiser med dioder. Antal per kanal?
\item To lydindgange. Type?
\item Mixer, som gør det muligt at blande lyden fra de to indgange
\item Mono
\item Indgangsvælger på selve forstærkeren
\item Digital volumenkontrol med 10 niveauer
\item 7 segment display til visning af volumen
\item Kortslutningssikring?
\item Klasse?
\end{enumerate}

\section*{Fjernbetjening}
\begin{enumerate}
\item IR
\item Volumen
\item Indgangsvælger
\item Tænd/sluk
\item Rækkevidde?
\end{enumerate}