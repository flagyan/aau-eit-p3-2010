Vi startede ud med at beslutte os for, at måden vi skulle dæmpe signalet på, var at trække det til stel, før det blev summeret i summationsforstærkeren. Dette gave os at vi skulle have R3 og R4, for at kunne dæmpe signalet, uden at trække det endelige output til stel. Outputtet fra en gate er 5V, hvilket giver os V2. Der skal desuden løbe en basis strøm I_B, hvilket giver os modstanden R5. C1 er der for kun at få signalet over i indgangsvælger trinet. Vi fandt dog rimeligt hurtigt ud af at et spændingsving omkring 0V ville give prolemer med strøm løbende fra emitteren til collectoren i Q1, når signalet trækkes under 0V. Derfor blev vi enige om at lave et DC offset, hvilket vi gjorde med R1 og R2 samt V1. R1 og R2 er valgt til 100k hver, hvilket giver en spændingsdeling på 1:2 imellem dem, hvilket ved 15V giver et DC offset på 7.5V. R3 og R4 er valgt efter at have en indgangsimpedans over 22, både når indgangen er tændt og slukket. For at regne på dette AC mæssigt, kortsluttes kondensatorerne, hvilket betyder at R1 og R2 begge går til stel. Derfor sidder de i parallel, sammen med Re, som består af R3 og R4 i serie. Ved et slukket signal er der ideelt stel mellem R3 og R4, derfor er R4 ligegyldig her. For at regne R3 opstilles der en parallelkoblingsformel:
\begin{equation}
\frac{1}{\frac{1}{100}+\frac{1}{100}+\frac{1}{R}}=22
R=39.29
\end{equation}
Dette giver altså en R på ca. 40k, når signalet skal slukkes, hvilket altså vil sige R3. Når signalet så er tændt igen, er indgangsimpedansen højere, da R3 og R4 skal adderes. Vi har valgt en R4 på ca 7k, hvilket giver en total indgangsimpedans på:
\begin{equation}
\frac{1}{\frac{1}{100}+\frac{1}{100}+\frac{1}{40.2+7.32}}=24.36
\end{equation}