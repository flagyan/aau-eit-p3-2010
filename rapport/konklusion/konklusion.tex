\chapter{Konklusion}
\label{konklusion}
Målet i projektet er at fremstille en berøringsfri skuffe, der kan reducere antallet af sygehuserhvervede infektioner på hospitalerne. Det er lykkedes at udvikle en prototype på en berøringsfri skuffe, der består af et RFID-modul, et virtuelt håndtag, en afstandssensor, en DC-motor og en microcontroller der binder alle modulerne sammen.

Skuffen er afprøvet i en accepttest, se kapitel \ref{acceptest}, og lever op til 4 ud af 7 testede krav, hvilket vurderes som tilfredsstillende for dette projekt.

Protypen er i stand til at køre frem og tilbage uden berøring. Den kan kun låses op med et godkendt RFID-tag og hvis den bliver efterladt åben, lukker den selv i. Når skuffen har været efterladt i 31 sek. lukker den i og låser sig selv, hvilket er 1 sek. over det krav, der er stillet til skuffen. Det ene sekunds overskridelse vurderes som værende ligegyldigt, da hovedfunktionaliteten fra kravet er til stede. Skuffen lever ikke op til kravet om, at den skal kunne stoppe indenfor 1 cm efter den er blevet sluppet. I praksis er det ikke noget, som har betydning for om skuffens andre funktionaliteter fungerer eller ej. Dertil vurderes det, at kravet ville kunne efterleves, hvis der var mere tid til at forfine programmeringen af microcontrollerens software.

Prototypen af skuffen vurderes ikke som klar til afprøvning på et hospital, men mere som et $"$proof of concept$"$. Med lidt mere udvikling og programmering af skuffens software, vurderes det også som realistisk, at skuffen kan reducere antallet af sygehuserhvervede infektioner på hospitalerne. Derudover kan skuffen, i dens nuværende form, spare personalet for en afspritning hver gang de skal i skuffen, da de ikke længere behøver at spritte deres hænder af efter de har rørt ved skuffehåndtaget. En uddybende salgsvurdering er beskrevet i appendiks A.