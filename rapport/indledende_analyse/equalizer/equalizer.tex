\section{Equalizer}
\label{equalizer}
Det menneskellige øre kan opfatte frekvenser fra ca. 20-20k Hz\fixme{kilde: http://www.hoerelse.info/page.dsp?page=414}. Dette sætter en naturligt bredde for frekvensbåndet, forstærkeren skal kunne operere indenfor. Udover at en given elektrisk komponent ikke vil være ens over hele frekvensbåndet, vil det akustiske miljø samt højtalerne også have indflydelse på den endelige oplevelse.  Derfor kan det være nødvendigt at regulere på de forskellige frekvenser, for at opnå den ønskede lyd. En equalizer benyttes til at dæmpe de forskellige frekvensbånd, i forhold til hinanden. En equalizer i en forstærker vil ofte være bredspektret og blive benyttet til at korrigere mere generelle ændringer i lyden. Hvis brugeren ønsker mere specifikke indstillinger, vil en dedikeret equalizer ofte benyttes. Da frekvensbåndet det menneskelige øre kan høre består af præcis 3 dekader, inddeles frekvensbåndene i equalizeren efter disse:
%Udtrykket equalizer stammer fra den originale hensigt med opfindelsen; at få det optagede til at lyde som den originale kilde. Dette gøres bl.a. for at kompensere for unøjagtigheder i optagelsesudstyr. Ved at dæmpe og forstærke individuelle frekvensbånd, er det muligt at få præcis den lyd brugeren kunne tænke sig. 
%En equalizer i en forstærker vil ofte være bredspektret og blive benyttet til at kompensere for det akustiske miljø brugeren befinder sig i, samt unøjagtigheder som følge af de benyttede komponenter. Hvis man har brug for mere specifikke indstillinger vil benyttet en dedikeret equalizer. Derfor har projektgruppen valgt at have 3 frekvensbånd\fixme{kilde: pdf dokument.}.

\begin{itemize}
\item Low: 20 - 200 Hz
\item Mid: 200 - 2000 Hz
\item High: 2000 - 20000 Hz
\end{itemize}

%Frekvensbåndene strækker sig fra 20 Hz til 20 kHz, da dette er det maksimale frekvensbånd det menneskelige øre kan opfatte\fixme{kilde: http://www.hoerelse.info/page.dsp?page=414}. Da dette frekvensbånd består af præcis 3 dekader, har projektgruppen valgt at inddele equalizerens frekvensbånd i disse.
%Hvor mange indstillingmuligheder der er på en equalizer, afhænger af antal af frekvensbånd, som kan indstilles uafhængigt af hinanden. Der opstilles derfor en række frekvensbånd, som et mål for projektet:
%Idéen med en equalizer er at kunne justere på styrken af de forskellige frekvenser i et signal, uafhængigt af hinanden. Dette benyttes til at få præcis den lyd, som brugeren ønsker. Eksempelvis kunne en bruger vælge at skrue op for de lave frekvenser, for at gøre bassen i et signal mere dominerende. I praksis er det dog en kombination af at forstærke nogle frekvensbånd, og dæmpe andre, på tværs af hele det hørbare område, for at skabe nøjagtigt den lyd der ønskes.\fixme{Skriv om} Dette er en hel videnskab i sig selv, men for at gøre mulighederne for dette så store som muligt, er det smart\fixme{andet ord?} at have et bredt udvalg af justerbare bånd. Dette gøres, analogt, ved hjælp af forskellige båndpas filtre.

\subsection{Visualizer}
\label{visualizer}
Visualizeren benyttes til at illustrere styrken af de signalerne i de forskellige frekvensbånd. I teorien kan en analog visualizer have uendeligt stor opløsning. I dette projekt vil det dog ikke give mening, ud fra et læringsmæssigt standpunkt at lave for stor opløsning, da dette bare er gentagelse af de samme basale elementer. Derimod vil en for lav opløsning heller ikke kunne bruges til noget. Derfor er en opløsning på seks dioder pr. frekvensbånd valgt. Dette giver desuden mulighed for at vise signalstyrken med farver: 2 grønne, efterfulgt af 2 gule, efterfulgt af 2 røde dioder.
%En visualizer giver et visuelt udtryk for, hvordan equalizeren er indstillet. Den angiver lydniveauet for hvert frekvensbånd, equalizeren dækker over. Dette giver bl.a. brugeren mulighed for at se hvilke frekvenser, der vil være optimale at justere på, for at få det ønskede output.