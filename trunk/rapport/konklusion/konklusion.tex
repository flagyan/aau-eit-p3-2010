\chapter{Konklusion}
\label{konklusion}

Formålet med dette projekt var at designe en HiFi-forstærker med digital styring. Målet var at designe en forstærker med to indgange, mikrofon og liniesignal, forforstærker, indgangsvælger, volumenkontrol, equalizer, visualizer og effektforstærker. Disse moduler skulle leve op til kravspecifikationen, som består af gængse standarder og krav bestemt af projektgruppen. 
Alle moduler undtagen tonekontrollen blev designet og simuleret. Equalizer og visualizer blev udeladt på grund af tidsmangel. Kortslutningssikringen er det eneste modul som ikke blev bygget og testet grundet at det indledende design var fejlbehæftet og tidsmangel. De resterende moduler er bygget og testet, samt blevet vurderet med udgangspunkt i de opsatte krav. 
Forforstærkeren og indgangsvælgeren bestod alle de opstillede krav. Volumenkontrollen var fejlbehæftet da tælleren ikke var konsistent i op og nedtælling samt at den har 52 trin, hvor kravet var 51. 
Effektforstærkeren overholdt alle kravene bortset fra at den dæmpede for meget indenfor frekvensområdet 20 Hz - 20 kHz. 

Efter at alle moduler var blevet testet hver for sig blev HiFi-forstærkeren implementeret og testet. Testene på det samlede system viste at ingen af de opsatte krav blev overholdt. 