\chapter{Konklusion}
\label{konklusion}
Temaet for dette semester har været $"$Analog og digital elektronik$"$ og temaet for selve projektet har været $"$High Fidelity (Hi-Fi) forstærker med digital styring$"$. For at leve op til semestertemaet valgte gruppen fra starten at designe mest muligt med diskret elektronik fremfor integreret elektronik, dog viste tidsmangel sig som en faktor, der kom til at betyde, at dette ikke altid kunne realiseres.\\\\
Projektets HiFi-forstærker blev valgt til at skulle bestå af følgende syv elementer; frontpanel, forforstærker, indgangsvælger, equalizer, visualizer, volumenkontrol og effektforstærker. Kravene til disse elementer blev opstillet ud fra en undersøgelse af gældende aktuelle standarder samt projektgruppens ønsker, hvilket er dokumenteret i kapitel \ref{valgafloesning}. Efter kravspecifikationen kom på plads, blev de enkelte elementer designet med baggrund i viden gruppen har tilegnet sig gennem semestres kurser. Hver element blev desuden simuleret for at verificere funktionaliteten inden elementet blev bygget. Equalizeren og visualizeren blev dog aldrig hverken designet, simuleret eller bygget på grund af tidsmangel. Dog er tankerne bag deres funktion ikke fjernet fra denne rapport, da projektgruppen har valgt at opbygge rapporten så den beskriver projektprocessen, fremfor at sælge produktet. Desuden blev der, også grundet tidsmangel, aldrig fremstillet et reelt frontpanel, men de nødvendige interaktionsmuligheder er derimod placeret på hvert de aktuelle elementer.\\\\
De resterende fire elementer blev bygget, hvorefter de blev testet enkeltvis for, at vurdere hvorvidt disse levede op til kravene opstillet specifikt hvert element. Som dokumenteret i afsnittene kaldet $"$accepttest$"$ til hvert element, varierer graden af succes fra element til element. Forforstærkeren lever op til alle kravene stillet specifikt til den, hvilket indgangsvælgeren også gør. Volumenkontrollen derimod lever op til alle pånær et enkelt, hvilket dog snyder i forhold til brugeroplevelsen, da ingen af kravene dækker dette område. Effektforstærkeren klarer i testen fire ud af syv specifikke krav, hvilket der, lige som tilfældet er med volumenkontrollen, ikke har været tid til at forbedre. Det er dog ganske klart hvorfor effektforstærkeren ikke lever op til et af kravene, nemlig kravet om den afsatte effekt, hvilket lyder på minimum 20 W i en 8 \ohm 's belastning. Dette opfyldes ikke, hvilket er forventeligt da hele designet bygger på at opnå præcis 20 W, hvilket derfor ved korrekt design udelukker en større effekt. Dette er en fejl i designprocessen, som stod klar for projektgruppen tidligt i processen, uden der dog blev valgt at ændre på det, da det ikke har nogen læringsmæssig fordel. Desuden blev der til effektforstærkeren designet en kortslutningssikring, som, grundet tidsmangel, aldrig blev implementeret og testet. Denne blev dog simuleret og årsagerne til dens teoretiske funktionalitetsfejl blev dokumenteret.\\\\
Med hvert element opbygget og testet, blev elementerne sat sammen til en endelig test af hele HiFi-forstærkeren. Resultateterne af denne test viste, at HiFi-forstærkeren lever op til lige over halvdelen af de oprindeligt opstillede krav til hele systemet. Nogle af resultaterne viser at den samlede HiFi-forstærker ikke lever op til nogle krav, som de enkelte elementer ellers lever op til. Årsagerne til disse forskelle har der ikke været tid til at undersøge, i et forsøg på at udbedre.

%Formålet med dette projekt var at designe en HiFi-forstærker med digital styring. Målet var at designe en forstærker med to indgange, mikrofon og liniesignal, forforstærker, indgangsvælger, volumenkontrol, equalizer, visualizer og effektforstærker. Disse moduler skulle leve op til kravspecifikationen, som består af gængse standarder og krav bestemt af projektgruppen. 
%Alle moduler undtagen tonekontrollen blev designet og simuleret. Equalizer og visualizer blev udeladt på grund af tidsmangel. Kortslutningssikringen er det eneste modul som ikke blev bygget og testet grundet at det indledende design var fejlbehæftet og tidsmangel. De resterende moduler er bygget og testet, samt blevet vurderet med udgangspunkt i de opsatte krav. 
%Forforstærkeren og indgangsvælgeren bestod alle de opstillede krav. Volumenkontrollen var fejlbehæftet da tælleren ikke var konsistent i op og nedtælling samt at den har 52 trin, hvor kravet var 51. 
%Effektforstærkeren overholdt alle kravene bortset fra at den dæmpede for meget indenfor frekvensområdet 20 Hz - 20 kHz. 
%Efter at alle moduler var blevet testet hver for sig blev HiFi-forstærkeren implementeret og testet. Testene på det samlede system viste at ingen af de opsatte krav blev overholdt. Der er ikke blevet vurderet hvad der var skyld i dette pga. tidsmangel. 