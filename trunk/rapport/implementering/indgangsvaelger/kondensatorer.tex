


\subsubsection*{kondensatorer}



\begin{equation}
C = \frac{1}{2 \cdot \pi \cdot f \cdot R}
\end{equation}

\begin{equation}
f_n = \frac{20 Hz}{10} = 2 Hz
\end{equation}

\begin{equation}
R_{C_1} = \frac{1}{\frac{1}{R_3} + \frac{1}{R_4} + \frac{1}{R_1 + R_2}} = \frac{1}{\frac{1}{100 k\ohm} + \frac{1}{100 k\ohm} + \frac{1}{40,2 k\ohm + 7,32 k\ohm}} = 24,4 k\ohm
\end{equation}

\begin{equation}
C_1 = \frac{1}{2 \cdot \pi \cdot f_n \cdot R_{C_1}} = \frac{1}{2 \cdot \pi \cdot 2 Hz \cdot 24,4 k\ohm} = 3,3 \mu F
\end{equation}

\begin{equation}
R_{C_2} = R_5 + R_L = 47,5 k\ohm + 47,5 k\ohm = 95,0 k\ohm
\end{equation}

\begin{equation}
C_2 = \frac{1}{2 \cdot \pi \cdot f_n \cdot R_{C_2} } = \frac{1}{2 \cdot \pi \cdot 2 Hz \cdot 95,0 k\ohm} = 838 nF
\end{equation}

\begin{equation}
R_{C_3} = R_L = 47,5 k\ohm
\end{equation}

\begin{equation}
C_3 = \frac{1}{2 \cdot \pi \cdot f_n \cdot R_{C_3} } = \frac{1}{2 \cdot \pi \cdot 2 Hz \cdot 47,5 k\ohm} = 1,7 \mu F
\end{equation}