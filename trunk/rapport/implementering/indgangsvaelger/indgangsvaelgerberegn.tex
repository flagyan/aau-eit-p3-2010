Modstandene $R_1$ og $R_2$ er begge valgt til 100k\ohm, for at give et DC-offset på ca 7.5V, halvdelen af V1. Dog bliver offsettet ikke helt disse 7.5V, da spændingsdelingen mellem $R_1$ og $R_2$ er en belastet spændingsdeler. Den faktiske spændingsdeling kan udregnes, ved at sætte en seriekobling af $R_3$ og $R_4$ parallelt med $R_2$, da disse går til virtuelt stel ved indgangen til opampen.
\begin{equation}
15V-15V*\frac{R_1}{R_1+\frac{1}{\frac{1}{R_2}+\frac{1}{R_3+R_4}}=4.62V
\end{equation}\fixme{der er noget galt her, men kan ikke lige se det, ud fra teksten bare}
Dog er DC-offsettet underordnet, så længe det ligger indenfor et område transistoren kan arbejde med, og $DC-offset - AC-Peakværdi > 0$. DC-offsettet alligevel bliver filtreret fra gennem en kondensator, før summationsforstærkeren, så det har ikke ind
Modstanden $R_3$ er valgt ud fra, at signalet skal se en indgangsimpedans på minimum 22K\ohm. Når transistoren er tændt er indgangsimpedansen mindst. $R_1$, $R_2$ og $R_3$ sidder stadig alle parallelt hvilket giver:
\begin{equation}
\frac{1}{\frac{1{}R_1}+\frac{1}{R_2}+\frac{1}{R_3}}=22K
R_3=39.29K
\end{equation}
Denne er så valgt til 40.2K, for at passe med E\fixme{Hvilken række er det det passer med?}rækken.
For at kunne afbryde de enkelte signaler, kan transistoren Q1 trække signalet til stel. for at tillade at hele signalet bliver trukket til stel, skal hele den strøm der løber igennem systemet føres ned igennem transistoren. Den maksimale strøm der kan løbe i systemet er summen af DC-strømmen og AC-strømmen. 
Først findes DC-strømmen. DC spændingen i kredsløbet er sat til 15V. Hvis transistoren en direkte kobling, så der ligger stel mellem $R_3$ og $R_4$, vil $R_2$ og $R_3$ sidde i parallel til stel. På denne måde finder vi den maksimale mængde strøm, der kan løbe igennem $R_3$. Vi kan så regne os frem til, at den største strøm der vil løbe igennem $R_3$, og derfor også transistoren, er ca. 0.08 mA
Den maksimale AC-strøm findes ved at kortslutte alle kondensatorer, og regne med det størst mulige AC-udsving. En konsekvens af, at alle kondensatorer bliver kortsluttet, er at forsyning og stel også bliver kortsluttet, da disse er forbundet med en kondensator\fixme{teoretisk? Eller faktisk?}. Dette giver at $R_1$, $R_2$ og $R_3$ alle sidder i parallel til stel. Ud fra dette findes AC-strømmen til at være ca. 0.05 mA. Dette giver en samlet max strøm på ca. 0.13 mA.
Strømmen $I_B$, som løber ind i basis på transistoren skal mindst være den samlede strøm, divideret med $H_FE$:
\begin{equation}
I_B = \frac{0.13 mA}{H_FE} = xx mA
\end{equation}
Da den styring vi benytter\fixme{mere præcist, lige nu ved jeg ikke præcist hvad vi bruger} outputter 5V, kan vi derfor finde $R_B$'s maksimale værdi:
\begin{equation}
R_B = \frac{5V}{I_B} = xx K\ohm
\end{equation}
Alt under dette vil give en større strøm ind i basis, hvilket vil give en større strøm ind i collectorbenet og vil derfor også kunne trække signalet til stel.

Modstanden $R_4$ har ikke nogen indvirkning på indgangsimpedansen, når transistoren er tændt og signalet derfor er slukket. Når signalet er tændt, sidder den i serie med $R_3$, hvilket giver en højere indgangsimpedans:
\begin{equation}
R_indgang=\frac{1}{\frac{1}{R_1}+\frac{1}{R_2}+\frac{1}{R_3+R_4}}
\end{equation}
