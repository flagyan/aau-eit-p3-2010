\documentclass[a4paper,11pt,fleqn,oneside, openright]{memoir} % Brug openright hvis chapters skal starte p� h�jresider; openany, oneside

%%%% PACKAGES %%%%

% �� Overs�ttelse og tegns�tning �� %
\usepackage[utf8]{inputenc}					% G�r det muligt at bruge �, � og � i sine .tex-filer
\usepackage[danish]{babel}							% Dansk sporg, f.eks. tabel, figur og kapitel
\usepackage[T1]{fontenc}								% Hj�lper med orddeling ved �, � og �. S�tter fontene til at %v�re ps-fonte, i stedet for bmp	
\usepackage{latexsym}										% LaTeX symboler
\usepackage{xcolor,ragged2e,fix-cm}			% Justering af elementer
\usepackage{pdfpages}										% G�r det muligt at inkludere pdf-dokumenter med kommandoen \includepdf[pages={x-y}]{fil.pdf}	
\pretolerance=2500 											% G�r det muligt at justre afstanden med ord (h�jt tal, mindre orddeling og mere space mellem ord)
\usepackage{ulem}                       % Gennemstregning af ord med koden \sout{}
\usepackage{fixltx2e}										% Retter forskellige bugs i LaTeX-kernen
\usepackage{gensymb}					%Tilføjer \celsius og \degree notation	
\usepackage{lineno}						%Bruges til at tilføje linjetal
																	
% �� Figurer og tabeller � floats  �� %
\usepackage{flafter}										% S�rger for at dine floats ikke optr�der i teksten f�r de er sat ind.
\usepackage{multirow}                		% Fletning af r�kker
\usepackage{hhline}                   	% Dobbelte horisontale linier
\usepackage{multicol}         	        % Fletning af kolonner
\usepackage{colortbl} 									% Mulig�re farver i tabeller
%\usepackage{float}												% G�r det muligt at placere figurer hvor du vil.   \begin{figure}[!h] % Will not be floating.
\usepackage{wrapfig}										% Inds�ttelse af figurer omsv�bt af tekst. \begin{wrapfigure}{Placering}{St�rrelse}
\usepackage{graphicx} 									% Pakke til jpeg/png billeder
\pdfoptionpdfminorversion=6							% Muligg�r inkludering af pdf dokumenter, af version 1.6 og h�jere
	
% �� Matematiske formler og maskinkode ��
\usepackage{amsmath,amssymb,stmaryrd} 	% Bedre matematik og ekstra fonte
\usepackage{textcomp}                 	% Adgang til tekstsymboler
\usepackage{mathtools}									% Udvidelse af amsmath-pakken. 
\usepackage{eso-pic}										% Tilf�j billedekommandoer p� hver side
\usepackage{lipsum}											% Dummy text \lipsum[..]
\usepackage{rsphrase}										% Kemi-pakke til RS-s�tninger
\usepackage[version=3]{mhchem} 					% Kemi-pakke til lettere notation af formler

% �� Referencer, bibtex og url'er �� %
\usepackage{url}												% Til at s�tte urler op med. Virker sammen med hyperref
\usepackage[danish]{varioref}						% Giver flere bedre mulighed for at lave krydshenvisninger
\usepackage{natbib}											% Litteraturliste med forfatter-�r og nummerede referencer
\usepackage{xr}													% Referencer til eksternt dokument med \externaldocument{<NAVN>}

% �� Floats �� %
%\let\newfloat\relax 										% Memoir har allerede defineret denne, men det g�r float pakken ogs�
%\usepackage{float}

\usepackage[footnote,draft,danish,silent,nomargin]{fixme}		% Inds�t rettelser og lignende med \fixme{...} Med final i stedet for draft, udl�ses en error 																															for hver fixme, der ikke er slettet, n�r rapporten bygges.

%%%% CUSTOM SETTINGS %%%%

% �� Marginer �� %
\setlrmarginsandblock{3.5cm}{2.5cm}{*}	% \setlrmarginsandblock{Indbinding}{Kant}{Ratio}
\setulmarginsandblock{2.5cm}{3.0cm}{*}	% \setulmarginsandblock{Top}{Bund}{Ratio}
\checkandfixthelayout 									% Laver forskellige beregninger og s�tter de almindelige l�ngder op til brug ikke memoir pakker

%	�� Afsnitsformatering �� %
\setlength{\parindent}{0mm}           	% St�rrelse af indryk
\setlength{\parskip}{4mm}          			% Afstand mellem afsnit ved brug af double Enter
\linespread{1,1}												% Linie afstand

% �� Litteraturlisten �� %
\bibpunct[,]{[}{]}{;}{a}{,}{,} 					% Definerer de 6 parametre ved Harvard henvisning (bl.a. parantestype og seperatortegn)
%\bibliographystyle{bibtex/harvard}			% Udseende af litteraturlisten. Ligner dk-apali - mvh Klein

% �� Indholdsfortegnelse �� %
\setsecnumdepth{subsection}		 					% Dybden af nummerede overkrifter (part/chapter/section/subsection)
\maxsecnumdepth{subsection}							% �ndring af dokumentklassens gr�nse for nummereringsdybde
\settocdepth{section} 								% Dybden af indholdsfortegnelsen

% �� Visuelle referencer �� %
\usepackage[colorlinks=true]{hyperref}			 	% Giver mulighed for at ens referencer bliver til klikbare hyperlinks. .. [colorlinks]{..}
\hypersetup{pdfborder = 0}							% Fjerner ramme omkring links i fx indholsfotegnelsen og ved kildehenvisninger ��
\hypersetup{														%	Ops�tning af farvede hyperlinks
    colorlinks = true,
    linkcolor = black,
    anchorcolor = black,
    citecolor = black,
    urlcolor = black
}

\definecolor{gray}{gray}{0.80}					% Definerer farven gr�

% �� Ops�tning af figur- og tabeltekst �� %
 	\captionnamefont{
 		\small\bfseries\itshape}						% Ops�tning af tekstdelen ("Figur" eller "Tabel")
  \captiontitlefont{\small}							% Ops�tning af nummerering
  \captiondelim{. }											% Seperator mellem nummerering og figurtekst
  \hangcaption													%	Venstrejusterer flere-liniers figurtekst under hinanden
  \captionwidth{\linewidth}							% Bredden af figurteksten
	\setlength{\belowcaptionskip}{10pt}		% Afstand under figurteksten
		
% �� Navngivning �� %
\addto\captionsdanish{
	\renewcommand\appendixname{Appendiks}
	\renewcommand\contentsname{Indholdsfortegnelse}	
	\renewcommand\appendixpagename{Appendiks}
%	\renewcommand\cftchaptername{\chaptername~}				% Skriver "Kapitel" foran kapitlerne i indholdsfortegnelsen
	\renewcommand\cftappendixname{\appendixname~}			% Skriver "Appendiks" foran bilagene i indholdsfortegnelsen
	\renewcommand\appendixtocname{Appendiks}
}

% �� Kapiteludssende �� %
\definecolor{numbercolor}{gray}{0.7}			% Definerer en farve til brug til kapiteludseende
\newif\ifchapternonum

\makechapterstyle{jenor}{									% Definerer kapiteludseende -->
  \renewcommand\printchaptername{}
  \renewcommand\printchapternum{}
  \renewcommand\printchapternonum{\chapternonumtrue}
  \renewcommand\chaptitlefont{\fontfamily{pbk}\fontseries{db}\fontshape{n}\fontsize{25}{35}\selectfont\raggedleft}
  \renewcommand\chapnumfont{\fontfamily{pbk}\fontseries{m}\fontshape{n}\fontsize{1in}{0in}\selectfont\color{numbercolor}}
  \renewcommand\printchaptertitle[1]{%
    \noindent
    \ifchapternonum
    \begin{tabularx}{\textwidth}{X}
    {\let\\\newline\chaptitlefont ##1\par} 
    \end{tabularx}
    \par\vskip-2.5mm\hrule
    \else
    \begin{tabularx}{\textwidth}{Xl}
    {\parbox[b]{\linewidth}{\chaptitlefont ##1}} & \raisebox{-15pt}{\chapnumfont \thechapter}
    \end{tabularx}
    \par\vskip2mm\hrule
    \fi
  }
}																						% <--

\chapterstyle{jenor}												% Valg af kapiteludseende - dette kan udskiftes efter �nske

% �� Sidehoved �� %

%\makepagestyle{custom}																				% Definerer sidehoved og sidefod - kan modificeres efter �nske -->
%\makepsmarks{custom}{																						
%\def\chaptermark##1{\markboth{\itshape\thechapter. ##1}{}}		% Henter kapitlet den p�g�ldende side h�rer under med kommandoen \leftmark. \itshape g�r teksten kursiv
%\def\sectionmark##1{\markright{\thesection. ##1}{}}					% Henter afsnittet den p�g�ldende side h�rer under med kommandoen \rightmark
%}																														% Sidetallet skrives med kommandoen \thepage	
%\makeevenhead{custom}{Gruppe B130}{}{\leftmark}							% Definerer lige siders sidehoved efter modellen \makeevenhead{Navn}{Venstre}{Center}{H�jre}
%\makeoddhead{custom}{\rightmark}{}{Aalborg Universitet}			% Definerer ulige siders sidehoved efter modellen \makeoddhead{Navn}{Venstre}{Center}{H�jre}
%\makeevenfoot{custom}{\thepage}{}{}													% Definerer lige siders sidefod efter modellen \makeevenfoot{Navn}{Venstre}{Center}{H�jre}
%\makeoddfoot{custom}{}{}{\thepage}														% Definerer ulige siders sidefod efter modellen \makeoddfoot{Navn}{Venstre}{Center}{H�jre}		
%\makeheadrule{custom}{\textwidth}{0.5pt}											% Tilf�jer en streg under sidehovedets indhold
%\makefootrule{custom}{\textwidth}{0.5pt}{1mm}								% Tilf�jer en streg under sidefodens indhold

%\copypagestyle{nychapter}{custom}														% F�lgende linier s�rger for, at sidefoden bibeholdes p� kapitlets f�rste side
%\makeoddhead{nychapter}{}{}{}
%\makeevenhead{nychapter}{}{}{}
%\makeheadrule{nychapter}{\textwidth}{0pt}
%\aliaspagestyle{chapter}{nychapter}													% <--

\pagestyle{plain}																							% Valg af sidehoved og sidefod

% �� Fjerner den vertikale afstand mellem listeopstillinger og punktopstillinger �� %
\let\olditemize=\itemize							
\def\itemize{\olditemize\setlength{\itemsep}{-1ex}}
\let\oldenumerate=\enumerate						
\def\enumerate{\oldenumerate\setlength{\itemsep}{-1ex}}

%%%% CUSTOM COMMANDS %%%%

% �� Billede hack �� %
\newcommand{\figur}[4]{
		\begin{figure}[H] \centering
			\includegraphics[width=#1\textwidth]{billeder/#2}
			\caption{#3}\label{#4}
		\end{figure} 
}

% �� Specielle tegn �� %
\newcommand{\grader}{^{\circ}C}
\newcommand{\gr}{^{\circ}}
\newcommand{\g}{\cdot}

% �� Promille-hack (\promille) �� %
\newcommand{\promille}{%
  \relax\ifmmode\promillezeichen
        \else\leavevmode\(\mathsurround=0pt\promillezeichen\)\fi}
\newcommand{\promillezeichen}{%
  \kern-.05em%
  \raise.5ex\hbox{\the\scriptfont0 0}%
  \kern-.15em/\kern-.15em
  \lower.25ex\hbox{\the\scriptfont0 00}}

%%%% ORDDELING %%%%

\hyphenation{hvad hvem hvor}

%%%% Tabeler %%%%
\usepackage{threeparttable}
\usepackage[tableposition=top]{caption}

%%%% listings %%%%
\usepackage{listings}
\lstset{language=C}
\lstset{backgroundcolor=,rulecolor=}
\lstset{commentstyle=\textit}
\usepackage{lastpage}

\usepackage{marvosym}

%%%% top/tail %%%%

%\newcommand{\toptail}[1]{\fboxsep2mm\fbox{\begin{minipage}{145.3mm}#1\end{minipage}}}

\newcommand{\Ohm}{\ohm}
\newcommand{\my}{\mu}
\newcommand{\kilde}[1]{\fixme{Kilde: #1}}

 \usepackage[electronic]{ifsym}

\begin{document}
\chapter{Måling på Monacor MCE-4000}
\label{mic_output}
Denne målerapport dokumenterer en måling på en Monacor MCE-4000 mikrofon foretaget i B3-209, Fredrik Bajers Vej 7. Målingen blev foretaget den 30. november 2010 af Benjamin Krebs og Jacob Hansen.
\section{Formål}
\label{mic_output_formaal}
Målingens formål er:
\begin{itemize}
\item At bestemme peakspændingen på udgangen fra en Monacor MCE-4000 mikrofon ved 70 dB(A), i en afstand af 0,1 m.
\item At bestemme peakspændingen på udgangen fra en Monacor MCE-4000 mikrofon ved 90 dB(A), i en afstand af 0,1 m.
\end{itemize}
\section{Testobjekt}
\label{mic_output_testobjekt}
Her skal testobjektet defineres entydigt. Der kan evt. henvises til diagrammer i rapporten.\\
Det er vigtigt, at alle målepunkter er veldefinerede.\\

\section{Teori}
\label{mic_output_teori}
Lydtrykket for almindelig tale er 60 dB(A) i en afstand af én meter\fixme{kilde: http://www.es.aau.dk/sections/acoustics/press/fakta/lidt\_om\_lyd/}. Da det ikke forventes at brugen af en mikrofon under sang foregår ved én meters  afstand, skal denne værdi regnes om før den kan give et realistisk billede af hvilket lydtryk mikrofonen bliver udsat for. De 60 dB(A) regnes først om til Pa ved udregningen i formel (\ref{equ:db-pa1})

\begin{equation}
\label{equ:db-pa1}
p_{\mathrm{60~dB(A)}} = 10^{\frac{L_p}{20}} \cdot p_{\mathrm{ref}} = 10^{\frac{60~dB(A)}{20}} \cdot 20 \cdot 10^{-6}~\mathrm{Pa} = 0,02~\mathrm{Pa}
\end{equation}

Denne værdi er altså ved én meters afstand. Omregningen af lydtrykket til 0,1 m fra kilden foretages som i udregningen i formel (\ref{equ:distance}).

\begin{equation}
\label{equ:distance}
p_2 = p_1 \cdot \frac{r_1}{r_2} = 0,02~\mathrm{Pa} \cdot \frac{1~\mathrm{m}}{0,1~\mathrm{m}} = 0,2~\mathrm{Pa}
\end{equation}

Lydtrykket, i Pa, bliver altså, når afstanden deles med ti, ti gange større. Da dB(A) er en logaritmisk skala bliver 60 dB(A) til 80 dB(A) når afstanden reduceres fra 1 m til 0,1 m. Som forklaret i starten af kapitel \ref{forforstaerker} bliver arbejdsområdet dermed fra 70 db(A) til 90 dB(A). De forventede peakspændinger på udgangen af mikrofonen kan derfor regnes ved formel (\ref{equ:vmicpeak}), hvor lydtrykket igen er omregnet til Pa.

\begin{equation}
\label{equ:vmicpeak}
\hat{V}_{\mathrm{microphone}} = p \cdot 5\frac{\mathrm{mV}}{\mathrm{Pa}}
\end{equation}

Dette giver en teoretisk minimums og maksimumspeakspænding på udgangen på henholdsvis 3,16 mV og 31,6 mV.

\section{Måleopstilling}
\label{mic_output_maaleopstilling}
Her skal vises en tegning over måleopstillingen, så man klart kan se, hvordan udstyret er tilsluttet.\\
Hvis der bruges flere forskellige tilslutninger under målingen, kan der vises flere forskellige opstillinger, eller der kan skrives en forklarende tekst.\\

\section{Anvendt udstyr}
\label{mic_output_anvendtudstyr}
Alt væsentligt udstyr skal beskrives entydigt.\\

\section{Måleprocedure}
\label{mic_output_maaleprocedure}
Her beskrives klart og entydigt, hvordan målingen er foretaget inkl. alle ikke indlysende indstillinger af apparater. Eks.: \\
1. Spændingsforsyningen tilsluttes og indstilles til 15 V (måles med voltmeteret) \\
2. Generatoren indstilles til at give en sinusspænding med en amplitude på 14 mV (måles med oscilloskopet) \\
3. ....\\

\section{Resultater}
\label{mic_output_resultater}
Nogle resultater kan med fordel flyttes (eller kopieres) til rapporten – husk henvisning \\
Ofte angives tabeller i målejournalen og grafer i rapporten \\
Brug tabeller – resultater blandet med tekst bliver rodet\\
Datafiler bør (desuden) vedlægges rapporten på en CD – husk henvisning\\
Præcis formulering er vigtig!!\\
Angiv enheder - DC, RMS, amplitude, eller spids-spids værdier?\\

\section{Måleusikkerheder}
\label{mic_output_maaleusikkerheder}
Her angives væsentlige fejlkilder og usikkerheder i.f.b. med målingen. \\
Principielt skal man medtage alle usikkerheder og lave en samlet usikkerhedsberegning, men oftest nævnes kun de mest væsentlige. \\
Det er vigtigt at forklare uoverensstemmelser mellem beregnede, simulerede og målte data, men det hører hjemme i hovedrapporten – ikke i målejournalen. \\
I rapporten kan man evt. henvise til usikkerheder beskrevet i målejournalen.\\
Typiske årsager til måleunøjagtighed:\\
Måleinstrumenter påvirker (belaster) måleobjektet\\
Aflæsningsunøjagtighed\\
Analoge (antikke) viserinstrumenter	\\
Oscilloscop-cursor (pas på støj i ”auto-peak-peak”)\\
Støj, 50 Hz (100 Hz) brum, switch-mode spændingsforsyninger m.v.\\
Instrumentets unøjagtighed: Se manualen! \\
Multimetre: Frekvensafhængig måleusikkerhed \\
Oscilloscop: Både horisontal (lille) og vertikal usikkerhed\\

\end{document}