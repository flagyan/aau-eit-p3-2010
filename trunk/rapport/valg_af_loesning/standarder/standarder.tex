\section{Standarder}
\label{standarder}
I dette afsnit bliver der taget udgangspunkt i gældende standarder fra International Electrotechnical Commitee (IEC) og Deutsches Institut f\"{u}r Normung (DIN). Målet med standarder er at opstille nogle normer for hvad produkter skal leve op til, hvilket gøres for at standardisere markedet sådan at produkter fra forskellige producenter kan arbejde sammen og ikke kun virker med produkter fra samme producent. Kravene opstillet i standarderne er ikke lovkrav, men derimod retningslinier. Det er dog i de færrestes interesse ikke at overholde standarderne.

Der findes mange standarder indenfor HiFi-forstærker-området, men ikke alle vil blive gennemgået. Der er istedet valgt tre standarder ud, som dækker det relevante for projektet. De tre standarder der arbejdes med er IEC581 Part 6, IEC61938 1. udgave og DIN 45500 normen. 

\subsection*{IEC581 Part 6 - Amplifiers}
\label{IEC581}
Standarden IEC581 har titlen $"$High fidelity audio equipment and systems; Minimum performance requirements$"$ og er fra 1979. I dette projekt er det valgt kun at anvende del 6 af standarden da kun denne del har relevans for projektet. Del 6 af standarden opstiller minimumskrav en HiFi-forstærker skal overholde \cite{IEC581-6}.

\subsubsection*{Forvrængning}

\begin{itemize}
\item En HiFi-forstærker må maksimalt forvrænge 1,0 \%
\item En forforstærker må maksimalt forvrænge 0,5 \%
\item En effektforstærker må maksimalt forvrænge 0,5 \%
\item Forforstærker og effektforstærker samlet må maksimalt forvrænge 0,7 \%
\item Disse værdier skal som minimum overholdes i frekvensområdet fra 40 Hz til 16 kHz. Dog ved forforstærker og effektforstærker samlet er det tilladt at outputtet falder med 3 dB i frekvensområderne fra 40 Hz til 63 Hz og 12,5 kHz til 16 kHz
\item Ydermere skal alle værdier af forvrængning være overholdt nominel udgangseffekt og indtil 26 dB lavere 
\end{itemize}
I projektet er der valgt at overholde standarderne om forvrængning. Kravet om den samlede forvrængning fastsættes i afsnit \ref{thd}. 

\subsubsection*{Udgangseffekt}
\begin{itemize}
\item Der skal minimum være et output på 10 W per udgangskanal og det skal overholde kravet om forvrængning.
\item Hvis forstærkeren har mere end én udgangskanal skal alle kanaler kunne levere minimum 10 W samtidig.
\item Forstærkeren skal kunne levere det maksimale output, uden at overskride forvrængningskravene, i mindst 10 minutter ved 1 kHz, med alle kanaler tændt og en temperatur mellem 15 \celsius~ og 35 \celsius.
\end{itemize}
I projektet er det valgt at tage udgangspunkt i standardens angivelser omkring udgangseffekt. Det endelige valg af udgangseffekt er beskrevet i afsnit \ref{valg_udgangseffekt}.

\subsubsection*{Frekvensområde}
\begin{itemize}
\item Frekvensområdet skal som minimum gå fra 40 Hz til 16 kHz.
\item Der må være en tolerance på $\pm$ 1,5 dB for signaler der ikke er kommet igennem en equalizer, målt ved 1 kHz.
\item Der må være en tolerance på $\pm$ 2 dB for signaler der er kommet igennem en equalizer, målt ved 1 kHz.
\end{itemize}
I projektet er der valgt at HiFi-forstærkeren skal overholde de tolerancer der er angivet i standarden. Bestemmelsen af frekvensområdet tager udgangspunkt i standarden og den menneskelige hørelse og foretages i afsnit \ref{equalizer}.

\subsubsection*{Isolering af signaler}
\begin{itemize}
\item Isolering mellem signaler skal mindst være 40 dB fra 250 Hz til 10 kHz
\item Ved 1 kHz skal isoleringen mindst være 50 dB
\end{itemize}
I projektet er der valgt at isoleringen af signaler skal være større end 50 dB.

\subsection*{IEC61938 1. udgave}
\label{IEC61938}
Standarden IEC61938 har titlen $"$Audio-, video- og audiovisuelle systemer - Indbyrdes forbindelser og matchende værdier - Foretrukne matchende analoge signalværdier$"$ og er fra 1997. Standarden der er brugt i rapporten er 1. udgave. Standarden er taget med fordi den opsætter krav til indgangene i en HiFi-forstærker\cite{IEC61938}. 

\subsubsection*{Liniesignaler} 
\begin{itemize}
\item Indgangsimpedansen som liniesignalet bliver koblet til skal være større eller lig med 22 k\ohm 
\item Liniesignalets peakspænding skal være mellem 0,2 V og 2 V
\item Udgangsimpedansen for liniesignalet skal højst være 2,2 k\ohm
\end{itemize}
I projektet tages der udgangspunkt i at liniesignalerne der arbejdes med overholder disse standarder.

\subsubsection*{Mikrofonindgang}
\begin{itemize}
\item Indgangsimpedansen skal være større eller lig med 5 k\ohm
\item Inputspændingen skal være mellem 0,8 mV og 200 mV
\end{itemize}
I projektet er det valgt at overholde standarden for mikrofonindgangen.

\subsection*{DIN 45500 normen}
\label{DIN45500}
DIN 45500 normens fulde titel er $"$Deutsches Institut f\"{u}r Normung 45500$"$. Denne norm gælder for audioudstyr og er brugt til at bestemme et krav til belastningsimpedansen

\subsubsection*{Belastningsimpedans}
\begin{itemize}
\item For højtalere skal belastningsimpedansen være enten 4 \ohm~eller 8 \ohm.
\item Tolerance på maksimum 20 \% i frekvensområdet fra 40 Hz til 16 kHz.
\end{itemize}
I projektet er der valgt at belastningsimpedansen skal være 8 \ohm, med en tolerance på maksimun 20\% i frekvensområdet.