\section{Indgangsimpedans}
\label{valg_indgangsimpedans}
Indgangsimpedansen er at opfatte som en impedans der, ud fra en almindelig spændingsdeling, reducerer indgangssignalet. Man er, med den begrundelse, interesseret i en stor indgangsimpedans. Den mindste tilladte størrelse af indgangsimpedansen for en HiFi-forstærker er i standard IEC61938-1 bestemt til 22 k\ohm~ for liniesignalsindgange, se afsnit \ref{standarder}. Da størrelsen af udgangsimpedansen samtidig er bestemt til maksimalt 2,2 k\ohm~ for en liniesignalsudgang, kan betydningen af indgangsimpedansens størrelse regnes som vist i udregningen i formel (\ref{equ:indgangsimpedans22}). 

\begin{equation}
\label{equ:indgangsimpedans22}
\frac{22~k\ohm}{22~k\ohm + 2,2~k\ohm} = 0,91
\end{equation}

Det ses af udregningen i formel (\ref{equ:indgangsimpedans22}), at en indgangsimpedans af størrelsen 22 k\ohm~ vil medføre et indgangssignal på 91 \% af det oprindelige signal, ved en udgangsimpedans på 2,2 k\ohm. Med en større indgangsimpedans vil en større del af det oprindelige signal blive indgangssignalet. Der vælges at overholde standarden, så kravet bliver at indgangsimpedansen skal være større end 22 k\ohm~i HiFi-forstærkeren.
