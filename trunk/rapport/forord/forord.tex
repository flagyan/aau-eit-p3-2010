\chapter*{Forord}
\label{forord}
Denne rapport dokumenterer et 3. semester projekt, udarbejdet i perioden fra 2. september 2010 til 21. december 2010. Projektet er udført af gruppe 311 på Elektronik og IT-ingeniør uddannelsen på Aalborg Universitet. Temaet for dette semester er $"$Analog og digital elektronik$"$ og gruppen valgte $"$High Fidelity (Hi-Fi) forstærker med digital styring$"$ som undertema. I løbet af semesteret modtager gruppen undervisning i form af PE- og SE-kurser, som bliver holdt af Institut for Elektroniske Systemer. Gruppen tilegner sig desuden viden gennem en fri studieaktivitet. Omtalte kurser er:

\begin{table}[h]
\centering
\begin{tabular}{c c}
\textbf{PE-kurser:} & \textbf{SE-kurser:} \\
Måleteknik & Quasistatiske elektriske og magnetiske felter \\
Analog elektronik & Beregningsteknik indenfor elektronikområdet 1 \\
Tilbagekoblingsteori 1 & Grundlæggende AC-kredsløbsteori \\
Basal digitalteknik & \\
\end{tabular}
\label{tab:kurser1}
\end{table}
\begin{table}[h]
\centering
\begin{tabular}{c}
\textbf{Fri studieaktivitet:} \\
PCB design og fabrikation \\
\end{tabular}
\label{tab:kurser2}
\end{table}
Gruppen har i løbet af projektet bestræbet sig på at opbygge mest muligt af løsningen analogt, fremfor digitalt, selvom det ikke altid er den teknisk bedste løsning. Der er i løbet af rapporten nævnt, når et element er udviklet specielt med dette for øje.\\\\
Gruppe 311 består af: \\\\\\\\\\
\begin{table}[h]
\centering
\begin{tabular}{c c c}
$\overline{~~~~~~Benjamin~Krebs~~~~~~}$ & $\overline{~~~~~~Frederik~Juul~~~~~~}$ & $\overline{~~~~~~Jacob~Hansen~~~~~~}$\\
\end{tabular}
\label{tab:gruppemedlemmer1}
\end{table} \\\\\\
\begin{table}[h]
\centering
\begin{tabular}{c c}
$\overline{~~~~~~Jesper~Knudsen~~~~~~}$ & $\overline{~~~~~~Jonas~Hansen~~~~~~}$\\
\end{tabular}
\label{tab:gruppemedlemmer2}
\end{table}