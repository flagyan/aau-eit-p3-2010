\chapter{Indledning}
\label{indledning}
Formålet med en forstærker er at forstærke et signal, til det niveau brugeren ønsker. I HiFi-forstærkerens tilfælde, drejer det sig specifikt om lyd signaler, som forstærkes til et, af brugeren defineret, niveau. Disse bruges både i input- og output-situationer. Eksempelvis vil en HiFi-forstærker blive brugt hvis man har behov for at forstærke et mikrofonsignal, før det bliver optaget. Et andet eksempel, som dækker den mest gængse forståelse og benyttelse af en HiFi-forstærker, er hvor forstærkeren bliver brugt til at forstærke et signal, før signalet når en højtaler. Formålet med forstærkeren er, at tillade en større effekt at blive afsat i højtaleren, uden at belastningen har nogen indflydelse på det originale signal. Dette er især praktisk hvis signalet kommer fra en svag kilde, med en meget lille effekt. En HiFi-forstærker skal ofte ikke kun håndtere et enkelt signal, men tit også blande signalerne sammen, samt have mulighed for at kunne skrue op og ned, for de forskellige frekvenser i signalet. 

Navnet $"$HiFi-forstærker$"$ kommer fra det engelske $"$High Fidelity$"$, hvilket oversat betyder høj nøjagtighed. Dette kommer af, at man med en HiFi-forstærker forsøger at opnå, det man kalder høj transparency; altså at det, udover styrken, er umuligt at se på signalet, at det har været igennem forstærkeren. Tidlige forsøg på forstærkere, $"$LoFi$"$-forstærkere, havde en lav transparency, hvilket betød, at der var meget forstærker-skabt støj på signalet. Der findes standarder for bl.a. transparency og effektforstærker, så der er en tydelig forskel på HiFi- og LoFi-forstærkere.

HiFi-forstærkeren findes i dag, i større eller mindre versioner, i, praktisk talt, alt elektronisk udstyr, der gengiver lyd. Derfor er de en integreret, ofte usynlig, del af hverdagen og kan findes overalt.