\section{Equalizer}
\label{equalizer}
Udtrykket equalizer stammer fra den originale hensigt med opfindelsen; at få det optagede til at lyde som den originale kilde. Dette gøres bl.a. for at kompensere for unøjagtigheder i optagelsesudstyr. Dog benyttes en equalizer, nu om dage, i høj grad kreativt.  Ved at dæmpe og forstærke individuelle frekvensbånd, er det muligt at få præcis den lyd man\fixme{brugeren?} kunne tænke sig. 

En equalizer på en forstærker vil ofte være meget bredspektret og blive benyttet til at kompensere for det akustiske miljø brugeren befinder sig i, samt unøjagtigheder som følge af de benyttede komponenter. Hvis man har brug for mere specifikke indstillinger vil benyttet en dedikeret equalizer. Derfor har projektgruppen valgt at have 3 frekvensbånd\fixme{kilde: pdf dokument.}:

\begin{itemize}
\item Low: 20 - 200 Hz
\item Mid: 200 - 2000 Hz
\item High: 2 - 20 KHz
\end{itemize}
%Hvor mange indstillingmuligheder der er på en equalizer, afhænger af antal af frekvensbånd, som kan indstilles uafhængigt af hinanden. Der opstilles derfor en række frekvensbånd, som et mål for projektet:
%Idéen med en equalizer er at kunne justere på styrken af de forskellige frekvenser i et signal, uafhængigt af hinanden. Dette benyttes til at få præcis den lyd, som brugeren ønsker. Eksempelvis kunne en bruger vælge at skrue op for de lave frekvenser, for at gøre bassen i et signal mere dominerende. I praksis er det dog en kombination af at forstærke nogle frekvensbånd, og dæmpe andre, på tværs af hele det hørbare område, for at skabe nøjagtigt den lyd der ønskes.\fixme{Skriv om} Dette er en hel videnskab i sig selv, men for at gøre mulighederne for dette så store som muligt, er det smart\fixme{andet ord?} at have et bredt udvalg af justerbare bånd. Dette gøres, analogt, ved hjælp af forskellige båndpas filtre.

\subsection{Visualizer}
En visualizer giver et visuelt udtryk for, hvordan equalizeren er indstillet. Den angiver lydniveauet for hvert frekvensbånd, equalizeren dækker over. Dette giver bl.a. brugeren mulighed for at se hvilke frekvenser, der vil være optimale at justere på, for at få det ønskede output.