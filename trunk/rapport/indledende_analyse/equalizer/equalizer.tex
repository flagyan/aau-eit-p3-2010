\section{Equalizer}
\label{equalizer}
Idéen med en equalizer er at kunne justere på styrken af de forskellige frekvenser i et signal, uafhængigt af hinanden. Dette benyttes til at få præcis den lyd, som brugeren ønsker. Eksempelvis kunne en bruger vælge at skrue op for de lave frekvenser, for at gøre bassen i et signal mere dominerende. I praksis er det dog en kombination af at forstærke nogle frekvensbånd, og dæmpe andre, på tværs af hele det hørbare område, for at skabe nøjagtigt den lyd der ønskes.\fixme{Skriv om} Dette er en hel videnskab i sig selv, men for at gøre mulighederne for dette så store som muligt, er det smart\fixme{andet ord?} at have et bredt udvalg af justerbare bånd. Dette gøres, analogt, ved hjælp af forskellige båndpas filtre.

\subsection{Visualizer}
En visualizer giver et visuelt udtryk for, hvordan equalizeren er indstillet. Den angiver lydniveauet for hvert frekvensbånd, equalizeren dækker over. Dette giver bl.a. brugeren en mulighed for at se om der er nogle frekvenser som rammer toppen af peakpeakstads\fixme{mangler et ord her. Det er sent, det er ikke min skyld; semikolon.}, og som derfor har behov for at blive nedjusteret. 