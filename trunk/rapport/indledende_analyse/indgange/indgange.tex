\section{Lydindgangstyper}
\label{indgange}
En HiFi-forstærker skal have noget input for at kunne spille. Der er flere forskellige typer af input, også kaldet indgange. Der er flere slags analoge lydindgange, med hver sine formål og fordele.

\subsubsection{Linie}
Linie indgangen fungerer ved at lyden overføres analogt i det hørbare frekvens område. Signalet er ikke kodet på nogen måde. \fixme{Kilde: IEC61938\_1.pdf}

\subsubsection{Mikrofon}
Da signalet fra en mikrofon normalt er meget svagt i forhold til liniesignalet, skal det forstærkes inden det kan sendes videre. Dette gøres med en mikrofonforforstærker, og det er det der adskiller mikrofonindgangen fra linieindgangen. Signalet vil efter forforstærkeren være sammenligneligt med linie signalet.\fixme{Kilde: IEC61938\_1.pdf}

\subsubsection{Grammofon}
En grammofonplade er optaget med det der hedder RIAA-forbetoning\fixme{Skriv om RIAA forbetoning/efterbetoning etc.}. Dette gøres blandt andet for at forbedre lyd kvaliteten. Det betyder at en grammofonforforstærker skal have RIAA-efterbetoning. Signalet vil efter forforstærkeren være sammenligneligt med linie signalet. \fixme{Kilde: IEC61938\_1.pdf}

%\subsection{Digital}
%Der er som hos de analoge også mange typer af digitale lydindgange. For de digitale lydindgange er der helt andre krav til f.eks. båndbrede og impedans. Dette skyldes at i stedet for at bruge 20 Hz - 20 kHz området til at sende lyden, bruges der meget højere frekvenser for at få de digitale signaler til at kunne overføre data hurtigere. Digitale indgange har indgangsimpedans omkring 75 \ohm, dette er også tilfældet for Dolby Digital og S/PDIF. \fixme{Kilde: IEC 60958 type II}

%\subsubsection{Dolby Digital}
%Dolby Digital dækker over flere indgangstyper, og er i bund og grund en encoding i stedet for en indgang. Det kan f.eks. overføres med både coxialle- og optiske-forbindelser. En af fordelene ved dette blandt andet at der kan overføres surround sound med mange kanaler. \fixme{Kilde: 20\_Dolby\_E.\_Standards.P.pdf}

%\subsubsection{S/PDIF}
%S/PDIF dækker over et sæt af specifikationer for digital overførsel og fysik form. Det er en standard udarbejdet af primært Sony og Phillips. Lige som Dolby Digital kan S/PDIF bruges med både coaxialle- og optiske-forbindelser.\fixme{Kilde: IEC 60958 type II}

\begin{table}[h]
\centering
\begin{tabular}{l|l|l}
\hline\hline
Indgang & Impedans & Frekvensgang \\
\hline\hline
Linie & $\geq$22 k\ohm~ & 40 Hz - 16 kHz $\pm$ 1,5 dB \fixme{Kilde: DIN45500.pdf} \\
\hline
Mikrofon & $\geq$22 k\ohm~ & 40 Hz - 16 kHz $\pm$ 1,5 dB \fixme{Kilde: DIN45500.pdf} \\
\hline
Grammofon & 47 k\ohm~... & RIAA-efterbetoning \\
%\hline
%Dolby Surround & Analog & $\geq$10 k\ohm~ & Flad... \\
%\hline
%Dolby Digital & Digital & 75 \ohm~ $^{[1]}$ & Flad... \\
%\hline
%S/PDIF & Digital & 75 \ohm~ $^{[1]}$ & Flad... \\
\hline\hline
\end{tabular}
\caption{Tabel over minimumskrav fra standarder.}
\label{tab:standarder_krav}
\end{table}

%\begin{itemize}
%\item[]{[1] Ved coaxialle forbindelser}
%\end{itemize}