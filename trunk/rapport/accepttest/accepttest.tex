\chapter{Samlet accepttest}
\label{acceptest}
Efter systemet implementeredes, blev der udført målinger, som vist i appendiks \ref{maaling_hifi}. Resultaterne af THD-målingerne viser, som det kan ses på figur \ref{maalerapport_final1} til figur \ref{maalerapport_final4}, desværre at kravene ikke er opfyldt for nogle af situationerne pånær stereo-indgangen med 200 mV. Resultaterne for frekvensgangen fra mikrofonindgangen viser, som det kan ses på figur \ref{maalerapport_final5} og figur \ref{maalerapport_final6}, at forstærkningen er for stor i frekvensområdet fra 20 Hz til 500 Hz, hvorefter den er for lav i resten af frekvensgangen. Forstærkningen skulle have været ca. 56 dB, hvilket der er en for stor afvigelse fra i forhold til kravene. Resultaterne for frekvensgangen fra stereo-indgangen viser, som det kan ses på figur \ref{maalerapport_final7} og figur \ref{maalerapport_final8}, at forstærkningen er omkring de ønskede 19 dB i frekvensområdet fra 200 Hz til 20 kHz. Dog falder forstærkningen mere end kravene tillader fra 20 Hz til 200 Hz. Resultaterne af dæmpningsmålingerne viser, som kan ses på figur \ref{maalerapport_final9} til \ref{maalerapport_final16}, at kravene ikke er opfyldt da dæmpningen skulle være minimum 50 dB for et slukket signal. Alle målingerne for dæmpning viser dæmpninger som svinger og som i større eller mindre dele af frekvensgangen fra 20 Hz til 20 kHz er under 50 dB.\\\\
Hvordan HiFi-forstærkeren lever op til de samlede krav, stillet i kravspecifikationen i tabel \ref{tab:kravspec}, kan ses i tabel \ref{tab:kravspec:accept}.

\begin{table}[h]
\centering
\begin{tabular}{l|r|l|r}
\hline\hline
Område & Krav & Betingelse(r) & Status \\
\hline\hline
\multicolumn{4}{c}{\textbf{Teknisk:}} \\\hline
Forstærkerklasse & AB & & \checkmark\\[4pt]
Total Harmonic & < 1 \% & & $\mathcal{X}$ \\
Distortion & & $\circ$ < 0,5 \% i forforstærker & \checkmark\\
& & $\circ$ < 0,5 \% i effektforstærker & $\mathcal{X}$\\
& & $\circ$ Begge i effektområde & \\
& & ~~~fra 0 til -26 dB & \\[4pt]
Frekvensgang & 20 Hz - 20 kHz & $\circ$ < 1,5 dB ved ref. 1 kHz & $\mathcal{X}$ \\
& & $\circ$ < 3 dB dæmpning & \\
& & ~~~fra 20 Hz til 63 Hz og  & $\mathcal{X}$ \\
& & ~~~fra 12,5 kHz til 20 kHz & $\mathcal{X}$ \\[4pt]
Indgangstyper & Linie og mikrofon & $\circ$ Med $"$Monacor & \checkmark \\
& & ~~~MCE-4000$"$ mikrofon & \\[4pt]
Antal trin i & 4 & & \checkmark\\
indgangsvælger & & & \\[4pt]
Dæmpning af slukket & > 50 dB & $\circ$ Ved 20 Hz - 20 kHz & $\mathcal{X}$ \\
indgangssignal & & & \\[4pt]
Indgangsimpedans i & > 22 k\ohm & & \checkmark \\
liniesignalsindgang & & &\\[4pt]
Indgangsimpedans i & > 5 k\ohm & & \checkmark \\
mikrofonsignalsindgang & & & \\[4pt]
Equalizer-bånd & 3 & & $\mathcal{X}$ \\[4pt]
Styring af volumen- & Digital & & \checkmark \\
kontrol & & &\\[4pt]
Dæmpningsområde i & 0 dB - 50 dB & $\circ$ 1 dB per niveau & \checkmark \\
volumenkontrol & & & \\[4pt]
Udgangseffekt & > 20 W & $\circ$ I 8~\ohm-højtaler & $\mathcal{X}$ \\[4pt]
Udgangssignaltype & Mono & & \checkmark \\[4pt]
Kortslutningsstrøm & 3 A & $\circ$ Som peakstrøm & $\mathcal{X}$ \\
i udgangen & & & \\\hline
\multicolumn{4}{c}{\textbf{Frontpanel (input):}} \\\hline
Indgangsvælger & Èn trykknap & & \checkmark\\[4pt]
Volumenkontrol & To trykknapper & & \checkmark \\[4pt]
Equalizer & Èn drejeknap pr. bånd & & $\mathcal{X}$ \\\hline
\multicolumn{4}{c}{\textbf{Frontpanel (output):}} \\\hline
Indgangsvælger & To lysdioder & $\circ$ Én per indgang & \checkmark\\[4pt]
Volumedisplay & To 7-segmenter & & \checkmark \\[4pt]
Visualizer & 6 lysdioder & $\circ$ 2 grønne, 2 gule, 2 røde & $\mathcal{X}$ \\
\hline\hline
\end{tabular}
\caption{Status af krav for hele systemet}
\label{tab:kravspec:accept}
\end{table}