\section{Standarder}
Formålet med dette afsnit er at beskrive forskellige standarder der er relevante for opbygningen af en HiFi-forstærker, for dernæst at det kan udmunde i en kravspecifikation på baggrund af de valgte standarder . Standarder gruppen har valgt at arbejde med er :

\begin{itemize}              
\item IEC61938-1             
\item DIN 45500 normen    
\end{itemize} 

Der er taget udgangspunkt i IEC61938-1. IEC61938-1 er sidst opdateret i 1-12-1997 \ref{??}\fixme{ref to standard IEC 61938-1}. DIN 45500 normen er taget med i projektet selvom den er forældet, sidst opdateret i 01-1973 \ref{??}\fixme{ref to standard  DIN 45500 normen}. Det er gjort fordi der ofte refereres til den i de nyere standarder. Disse standarder er blevet undersøgt for forskellige parametre, hvorefter de nyeste eller mest relevante data er blevet valgt ud til at arbejde videre med.
\newline
\newline
Tabel \ref{standarder_krav} angiver de parametre standarderne er undersøgt for, samt hvilken specifik standard der er valgt til hver parameter.
 
\begin{table}[h]
\centering
\begin{threeparttable}
\label{standarder_krav}
\begin{tabular}{l|l|l}
\hline\hline
Område & Minimumskrav & Standard \\
\hline\hline
Udgangseffekt&Min. 10W mono \textbf{*}& DIN45500 \\
&Min. 2x6W stereo \textbf{*}&\\										
\hline
Frekvens omrade&40Hz - 16000Hz&DIN45500\\
\hline
THD&Max 1\%\textbf{**} &DIN45500\\
&Max 0.7\%\textbf{***}&\\
\hline
Belastningsimpedans&&DIN45500\\
Højtaler&4 eller 8\ohm \textbf{****} &\\
Hovedtelefon&200 eller 400\ohm \textbf{****} &\\
\hline
Indgangssignal& 0.5V&IEC 61938-1\\
\hline
Indgangsimpedans&Min. 22k\ohm &IEC61938-1\\
\hline
Udgangsimpedans&Max. 2.2k\ohm &IEC61938-1\\
\hline
Udgangsspænding&Max. 2V&IEC61938-1\\
&Min. 0.2V&IEC61938-1\\
\hline\hline
\end{tabular}
\caption{Tabel over minimumskrav fra standarder.}
\begin{tablenotes}
\item \textbf{*} 1000 Hz sinus signal i 10 min. Ved en omgivelse temperatur på 35°C
\item \textbf{**} Forforstærker og effektforstærker. Målt i effektbåndbredde 40-12500 Hz,med en udgangseffekt på minimum 10 W
\item \textbf{***} Forforstærker eller effektforstærker. Målt i effektbåndbredde 40-12500 Hz,med en udgangseffekt på minimum 10 W
\item \textbf{****}Tolerance på 20\%
\end{tablenotes}
\end{threeparttable}
\end{table} 

Der er i dette afsnit blevet opstillet en række krav fra de undersøgte standarder. Disse data vil blive brugt videre i kravspecifikationen.