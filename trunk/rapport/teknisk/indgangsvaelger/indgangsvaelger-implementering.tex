\subsection{Implementering}
Under konstruktionen af indgangsvælgeren kunne det konstateres at en kondensator ikke umiddelbart var tilgængelig i hverken størrelsesordenen 8$\mu$F eller 33,6$\mu$F. Dette problem blev løst ved at benytte sig af reglen om at hvis to komponenter sidder i parallel kan deres admittans adderes. Admittansen er den reciprokke værdi af resistansen. Da resistansen for en kondensator er givet ved $\frac{1}{s\cdot C}$ bliver admittansen derfor $s\cdot C$. Derfor vil to kondensatorer der sidder i parallel agere som én større kondensator. Derfor blev to 4$\mu$F brugt i stedet for en enkelt 8$\mu$F og en 10$\mu$F og en 22$\mu$F blev benyttet i stedet for en 33.6$\mu$F. Da tolerancen på en elektrolyt kondensator har en tolerance på 20\%  dømmes dette til at være acceptabelt.
