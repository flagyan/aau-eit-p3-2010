\chapter{Indgangsvælger}
\label{indgangsvaelger}
Indgangsvælgerens opgave er at gøre brugeren i stand til at vælge mellem 3 liniesignaler i form af et forforstærket mikrofonsignal og et liniesignal i stereo. Indgangsvælgeren skal være i stand til at samle alle inputsignaler, som brugeren ønsker at høre, til ét udgangssignal. Den skal altså med andre ord være i stand til at slukke for det eller de signaler brugeren ikke ønsker at høre og lukke de andre signaler igennem uden at påvirke dem. \\
Der er i tabel \ref{tab:krav_indgangsvaelger} opstillet de relevante krav til indgangsvælgeren fra kravspecifikationen. Disse krav danner grundlag for designprocessen som beskrives i resten af dette kapitel.

\begin{table}[h]
\centering
\begin{tabular}{l|r}
\hline\hline
Område & Krav \\
\hline\hline
Antal trin i & 4 \\
indgangsvælgeren & \\[4pt]
Indgangsimpedans & > 22 k\ohm \\[4pt]
Total Harmonic & < 1 \% \\
Distortion & \\[4pt]
Frekvensgang & 20 Hz - 20 kHz \\[4pt]
Dæmpning af slukket & > 50 dB ved 1 kHz \\
indgangssignal & \\
\hline\hline
\end{tabular}
\caption{Krav til indgangsvælgeren}
\label{tab:krav_indgangsvaelger}
\end{table}



