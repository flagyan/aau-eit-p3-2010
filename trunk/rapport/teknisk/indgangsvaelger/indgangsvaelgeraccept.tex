\section{Accepttest}
Indgangsvælgerens to funktioner er, at enten sende signalet igennem, med så lav dæmpningen som muligt, eller at dæmpe så meget som muligt. Der er i kravspecifikationen opsat et krav om en dæmpning på minimum 50dB, når signalet er slukket. Dette er det krav, indgangsvælgeren er designet omkring og den væsentligste faktor til de valg der er foretaget. Dæmpningen når signalet er tændt er ikke væsentlig, udover at den skal være så lav som muligt. Dæmpningen vil kunne forstærkes op igen og vil ikke være af betydning for slutbrugerens oplevelse.\fixme{Kan vi finde et krav?}

Ved simulering findes dæmpningen, når signalet et slukket, til -94,3 dB\fixme{Wut?}, hvilket mere end opfylder kravet. Når signalet er tændt findes dæmpningen til dB\fixme{Værdi}, hvilket er acceptabelt.