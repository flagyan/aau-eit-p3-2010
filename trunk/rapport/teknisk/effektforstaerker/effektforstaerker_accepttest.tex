\section{Accepttest}
Der kan ses ud fra målingerne at frekvensgangen, ved de lave frekvenser, ikke er tilfredsstillende. Forstærkningen, kan på figur \ref{fig:apeff:frek2v} i appendiks G, ved 20 Hz aflæses til ca. 17 dB, mens forstærkningen ved 63 Hz aflæses til 18,7 dB. Dette er en forskel på 1,7 dB, hvilket er mere end de 0,75 dB den skulle være indenfor. Forstærkningen ved 1 kHz er aflæst til ca. 19,1. Forskellen til de 63 Hz er ca. 0,4 dB, hvilket er meget tæt på de krævede 0,375 dB; afvigelsen hér kan skyldes aflæsningsfejl. Forstærkningen ved 12 kHz aflæses til 19,1 dB, mens forstærkningen ved 20 kHz aflæses til 19,3 dB. Denne afvigelse på 0,2 dB er under kravet på 0,75 dB, hvilket derfor er acceptabelt. Der aflæses kun figur \ref{fig:apeff:frek2v}, da denne og figur \ref{fig:apeff:frek200mv} bedømmes til at være meget éns.

Denne afvigelse fra kravene skyldes sansynligvis at kondensatorudregningen for lavpasfilteret i tilbagekoblingen er forkert. Hvis kondensatoren havde været større, have polen været flyttet ned i frekvens, hvilket ville have sendt mindre af signalet tilbage til differensforstærkeren. Dette ville have betydet en lavere dæmpning af signalet, hvilket ville have gjort frekvensgangen acceptabel. Grundet tidsmangel er dette dog ikke gjort.

Der kan, ud fra figur \ref{fig:apeff:thd2v} i appendiks G, aflæses at THD ved fuld udstyring når et max på ca 0,48\%, ved 20 kHz. Dette er under de 0,5\% der er defineret som et krav og er derfor accepteret. På figur \ref{fig:apeff:thd200mv} kan der aflæses en maksimal THD, inden for 20 Hz til 20 kHz , ved 900 Hz til 0,068\%. Dette er også acceptabelt.