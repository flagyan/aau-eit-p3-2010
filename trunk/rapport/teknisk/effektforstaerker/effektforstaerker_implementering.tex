\section{Implementering}
\label{effektforstaerker_implementering}

Under implementeringen indstilledes potentiometret på $R_7$'s og $R_8$'s plads til de beregnede værdier. Da effektforstærkeren blev tændt og inputsignalet var 0 V, blev spændingsfaldet over $R_E$ målt hvorefter strømmen gennem den beregnedes med Ohms lov. Den fundne strøm er lig med hvilestrømmen i udgangstrinnet. Det måtte konkluderes at de beregnede værdier ikke passede korrekt, hvormed potentiometret på $R_8$'s plads justeredes, så spændingsfaldet over $R_E$ passede overens med, hvad det skulle være hvis strømmen gennem modstanden skulle være lig med 45 mA. Den justerede værdi for $R_8$ er ikke dokumenteret, da den ikke umiddelbart kan måles mens den er monteret i kredsløbet.

Efter at have justeret hvilestrømmen korrekt opstod endnu et problem. Ved implementeringen blev $V_\mathrm{BE}$-multiplierens transistor, Q1 på figur \ref{fig:vbemulti}, monteret for sig. Det viste sig at hvilestrømmen, som løber gennem darlingtontransistorerne ved 0 V inputsignal, afhang markant af Q1's og darlingtontransistorernes temperatur. Symptomerne viste sig efter at have indstillet spændingen over $V_\mathrm{BE}$-multiplieren så hvilestrømmen var 45 mA, som beregnet, hvorefter der blev påtrykt et inputsignal, således at darlingtontransistorerne blev varme. Efter at have slukket for inputsignalet og målt hvilestrømmen viste det sig at den var steget med over 100 mA. Problemet blev løst ved at montere Q1 på samme køleplade som darlingtontransistorerne således at deres temperatur følges ad. På denne måde følges deres temperaturafhængige transistorparametre bedre ad og udsvinget i hvilestrøm ved kold kontra varm tilstand falder markant. 