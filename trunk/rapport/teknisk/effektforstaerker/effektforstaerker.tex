\chapter{Effektforstærker}
\label{effektforstaerker}
Formålet med en effektforstærker er, som nævnt i kapitel \ref{systemopbygning}, at levere en strømforstærkning der gør det muligt at afsætte den ønskede effekt i belastningsmodstanden, altså højtaleren. Dette skal ske uden at der afsættes en stor effekt i selve effektforstærkeren, altså uden den bliver opvarmet unødigt, da denne effekt vil være spild. Effektforstærkeren skal desuden levere denne strømforstærkning uden signalet forvrænges for meget. \\
Alle kravene der er stillet til effektforstærkeren er opstillet i tabel \ref{tab:krav_effektforstaerker}.

\begin{table}[h]
\centering
\begin{tabular}{l|r}
\hline\hline
Område & Krav \\
\hline\hline
Klasse & AB \\[4pt]
Nyttevirkning & > 25 \%  \\[4pt]
Forvrængning & < 0,5 \% \\[4pt]
Udgangseffekt & > 20 W ved 2 V input \\[4pt]
Frekvensgang & $\pm$ 0,375 dB ved 20 Hz - 20 kHz \\
& $\pm$ 0,75 dB fra 20 Hz til 63 Hz \\
& $\pm$ 0,75 dB fra 12,5 kHz til 20 kHz \\[4pt]
Belastningsimpedans & 8 \ohm \\[4pt]
Udgangssignaltype & Mono \\[4pt]
Kortslutningsstrøm (peak) & 3 A \\
\hline\hline
\end{tabular}
\caption{Krav til effektforstærkeren}
\label{tab:krav_effektforstaerker}
\end{table}
