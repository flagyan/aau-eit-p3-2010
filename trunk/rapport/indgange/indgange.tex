\section{Lydindgangstyper}
\label{indgange}
En Hi-Fi forstærker skal have noget input for at kunne spille. Der er flere forskellige typer af input, også kaldet indgange. Alle lydindgange kan deles ind ind i to grupper analog og digital.

\subsection{Analog}
Der er flere slags analoge lydindgange, med hver sine formål og fordele.

\subsubsection{Linie}
Linie indgangen er en meget simpel indgang. Simpel på den måde at lyden overføres analogt i det hørbare frekvens område. Signalet er altså ikke kodet på nogen særlig måde. \fixme{Kilde: IEC61938\_1.pdf}

\subsubsection{Mikrofon}
Da signalet fra en mikrofon normalt er meget svagt i forhold til linie signalet, så skal det forstærkes inden det kan sendes videre. Dette gøres med en mikrofon forforstærker, og det er det der adskiller mikrofon indgangen fra linie indgangen. \fixme{Kilde: ??}

\subsubsection{Grammofon}
En grammofonplade er optaget med det der hedder RIAA-forbetoning. Dette gøres blandt andet for at forbedre lyd kvaliteten. Det betyder så at en grammofon forforstærker skal have RIAA-efterbetoning. \fixme{Kilde: ??}

\subsubsection{Dolby Surround}
Dolby Surround er en analog kodning af fire kanaler på to kabler. Det er altså muligt at få analogt surround sound.\fixme{Kilde: 2\_Surround\_Past.Present.pdf}

\subsection{Digital}
Der er som hos de analoge også mange typer af digitale lydindgange. For de digitale lydindgange er der helt andre krav til f.eks. båndbrede og impedans. Dette skyldes at i stedet for at bruge 20 Hz - 20 kHz området til at sende lyden, bruges der meget højere frekvenser for at få de digitale signaler til at kunne overføre data hurtigere. Digitale indgange har indgangsimpedans omkring 75\ohm, \fixme{Kilde: IEC 60958 type II} dette er også tilfældet for S/PDIF.

\subsubsection{Dolby Digital}
Dolby Digital dækker over flere indgangstyper, og er i bund og grund en encoding i stedet for en indgang. Det kan f.eks. overføres med både coxialle- og optiske-forbindelser. En af fordelene ved dette blandt andet at der kan overføres surround sound med mange kanaler. \fixme{Kilde: 20\_Dolby\_E.\_Standards.P.pdf}

\subsubsection{S/PDIF}
S/PDIF dækker over et sæt af specifikationer for digital overførsel og fysik form. Det er en standard udarbejdet af primært Sony og Phillips. Lige som Dolby Digital kan S/PDIF bruges med både coxialle- og optiske-forbindelser.\fixme{Kilde: IEC 60958 type II}