\section{Version Kontrol}
\begin{enumerate}
\item{Subversion(SVN) skal bruges til alt version kontrol, medlemmer skal selvfølgelig altid vide hvad adressen er.}
\item{LaTeX bruges til alle dokumenter for at have en standard og kompatibilitet for alle operativ systemer.}
\end{enumerate}
\subsection{Rettelser}
\begin{enumerate}
\item Alle rettelser skal foregå når dokumentets status er complete
\item Alle rettelser skal laves med \textbackslash fixme i latex og kommentar i kildekode
\item Rettelser må kun laves af personen som "ejer" filen på det tidspunkt
\end{enumerate}
\subsection{Fælles syntaks til P2 rapport}
Følgende regelsæt er blevet opsat for B212s P2 rapport

\begin{enumerate}
\item Alle referencer skal være med \textbackslash ref
\item Alle filnavne skal skrives med lille og ingen mellemrum, hvis der er behov for mellemrum så skal det erstattes med et $"$-$"$, f.eks rapport-udkast.tex
\item Alle fil navne med dato skal skrives navn-dato-måned, f.eks dagsorden-16-10.tex
references, footnotes
\item Formelbogstavet for spænding er U.
\item Hvis en kilde angiver noget specifikt skal det stå ved anvendelsen. Hvis kilden er brugt til afsnittet skal den stå efter afsnittets afslutning.
\item Ohm skal skrives $\Omega$
\item Følgende forkortelser skal bruges
\begin{itemize}
\item for eksempel skal skrives med f.eks.
\item blandt andet skal skrives med bl.a.
\item og så videre skal skrives med osv. 
\item med hensyn til skal skrives med mht.
\item Millioner skrives som mio.
\item Milliarder skrives som mia.
\end{itemize}
\item Tjek for $"$man$"$ og $"$vi$"$ efter du har skrevet et afsnit
\item microcontroller skrives med c
\end{enumerate}

