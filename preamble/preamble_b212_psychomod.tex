\documentclass[a4paper,11pt,fleqn,twoside, openany]{memoir} 
% twoside: indryk af indhold så der er plads til samlingen i "ryggen"
% Openany: Chapters må starte på næste tomme side. Brug Openright for kun at starte på højreside.
\usepackage{color}
%%%%% PAKKER %%%%%

% Tegnsætning og sprog %

\usepackage[utf8]{inputenc}	
% Understøttelse af ÆØÅ

\usepackage[danish]{babel}
% Dansk sprog med dansk orddeling

\usepackage{xcolor,ragged2e,fix-cm}
% xcolor: mulighed for at indsætte farver, skygger osv.
% ragged2e: en rettelsespakke til funktionerne: flush, centering og raggedright
% fix-cm: Forbedrer skrifttyper

\pdfoptionpdfminorversion=6
\usepackage{pdfpages}
% Gør det muligt at inkludere pdf-filer med: \includepdf[pages={x-y}]{fil.pdf}

\pretolerance=2500
% Angiver afstanden mellem ord. Et højt tal giver hyppig orddeling samt større afstand mellem ord.

\usepackage{ulem}
% Gennemstregning af ord med koden \sout{}

\usepackage{fixltx2e}
% Retter forskellige bugs i LaTeX-kernen

\usepackage{gensymb}
% Tilføjer \celsius og \degree notation	

\usepackage{lineno}
% Bruges til at tilføje linjetal med: 
% \setpagewiselinenumbers
% \modulolinenumbers[5]
% \linenumbers


% Figurer, tabeller og floats %


\usepackage{flafter}
% Sørger for at figurer, floats, bliver så vidt muligt indsat dér hvor referencen til dem er skrevet

\usepackage{multirow}
% Gør det muligt at lave flere linjer i en tabel med funktionen: \multirow{num_rows}{width}{contents}

\usepackage{hhline}
% Dobbelte horisontale linier i tabeller etc.

\usepackage{multicol}
% Giver mulighed for flere kolonner

\usepackage{colortbl}
% Farver i tabeller med: \columncolor[rgb]{farve}[y][z] eller \rowcolor[rgb]{farve}[y][z]

\usepackage{wrapfig}
% Wrappe tekst om en figur

\usepackage{graphicx}
% Understøttelse af JPEG/PNG


% Matematik %

\usepackage{amsmath,amssymb,stmaryrd} 	
% Bedre matematik og ekstra fonte

\usepackage{textcomp}                 	
% Adgang til tekstsymboler

\usepackage{mathtools}									
% Udvidelse af amsmath-pakken. 


% FixMe %
\usepackage[footnote,draft,danish,silent,nomargin]{fixme}		
% Indsæt rettelser og lignende med \fixme{...} 
% Med final i stedet for draft, udløses en error for hver fixme, der ikke er slettet, når rapporten bygges.


% Referencer, bibtex og url'er %

\usepackage{url}												
% Til at sætte urler op med. Virker sammen med hyperref

\usepackage[danish]{varioref}
% Giver flere bedre mulighed for at lave krydshenvisninger

\usepackage{natbib}											
% Litteraturliste med forfatter-år og nummerede referencer

\usepackage{xr}													
% Referencer til eksternt dokument med \externaldocument{<NAVN>}

\bibpunct[,]{[}{]}{;}{a}{,}{,} 					
% Definerer de 6 parametre ved Harvard henvisning (bl.a. parantestype og seperatortegn)

%%%%% SIDEOPSÆTNING %%%%%



% Tabel- og figuropsætning %

\captionnamefont{
 		\small\bfseries\itshape}						
 		% Opsætning af tekstdelen ("Figur" eller "Tabel")
 		
  \captiontitlefont{\small}							
  % Opsætning af nummerering
  
  \captiondelim{. }											
  % Seperator mellem nummerering og figurtekst
  
  \hangcaption													
  %	Venstrejusterer flere-liniers figurtekst under hinanden
  
  \captionwidth{\linewidth}							
  % Bredden af figurteksten
  
	\setlength{\belowcaptionskip}{10pt}		
	% Afstand under figurteksten


% Indholdsfortegnelse %

\addto\captionsdanish{
	\renewcommand\appendixname{Bilag}
	\renewcommand\contentsname{Indholdsfortegnelse}	
	\renewcommand\appendixpagename{Bilag}
	\renewcommand\cftchaptername{\chaptername~}					
	% Skriver "Kapitel" foran kapitlerne i indholdsfortegnelsen
	\renewcommand\cftappendixname{\appendixname~}		
	% Skriver "Bilag" foran bilagene i indholdsfortegnelsen
	\renewcommand\appendixtocname{Bilag}
}


\setsecnumdepth{subsection}		 					
% Dybden af nummerede overkrifter (part/chapter/section/subsection)

\maxsecnumdepth{subsection}							
% Ændring af dokumentklassens grænse for nummereringsdybde

\settocdepth{subsection} 
% Dybden af indholdsfortegnelsen




% Kapitel udseende %

%\definecolor{numbercolor}{blue}{0.7}
\newif\ifchapternonum

\makechapterstyle{jenor}{									
  \renewcommand\printchaptername{}
  \renewcommand\printchapternum{}
  \renewcommand\printchapternonum{\chapternonumtrue}
  \renewcommand\chaptitlefont{\fontfamily{pbk}\fontseries{db}\fontshape{n}\fontsize{25}{35}\selectfont\raggedleft}
  \renewcommand\chapnumfont{\fontfamily{pbk}\fontseries{m}\fontshape{n}\fontsize{1in}{0in}\selectfont\color{white}}
  \renewcommand\printchaptertitle[1]{%
    \noindent
    \ifchapternonum
    \begin{tabularx}{\textwidth}{X}
    {\let\\\newline\chaptitlefont ##1\par} 
    \end{tabularx}
    \par\vskip-2.5mm\hrule
    \else
    \begin{tabularx}{\textwidth}{Xl}
    {\parbox[b]{\linewidth}{\chaptitlefont ##1}} & \raisebox{-15pt}{\chapnumfont \thechapter}
    \end{tabularx}
    \par\vskip2mm\hrule
		\fi
 	}
}


\chapterstyle{jenor}


% Andet %

\setlength{\parindent}{0mm}           	
% Størrelse af indryk

\setlength{\parskip}{4mm}          			
% Afstand mellem afsnit ved brug af dobbel-enter

\linespread{1,1}
% Linjeafstand


% Tabeller %

\usepackage{threeparttable}
\usepackage[tableposition=top]{caption}


% listings %

\usepackage{listings}
\lstset{language=C}
\lstset{backgroundcolor=,rulecolor=}
\lstset{commentstyle=\textit}
\usepackage{lastpage}