\documentclass{beamer}
\usetheme{Goettingen}
\usepackage[danish]{babel}
\usepackage[utf8]{inputenc}
\usepackage{times}
\usepackage{pstricks}
\usepackage{xmpmulti}
\usepackage{multimedia}
\usepackage{amsmath,amssymb}

\begin{document}

%%%%%%%%%%%%%%%%%%%%%%%%%%%%%% MODUL 1 %%%%%%%%%%%%%%%%%%%%%%%%%%%%%%

\section{Modul 1}
\begin{frame}{Modul 1}
\begin{itemize}
\item Hvad er kravene til RFID?
\begin{itemize}
\item Læse RFID tags
\item Undgå uønsket aktivering
\item \textbf{Lås}
\end{itemize}

\item Parallax RFID Card Reader er valgt
\begin{itemize}
\item Aktivering af RFID
\item \textbf{Låse skuffen}
\item Seriel interface til microcontrolleren
\item \textbf{Serielt kommunikation med RS232 syntax og TTL-niveauer}
\end{itemize}

\end{itemize}
\end{frame}

%%%%%%%%%%%%%%%%%%%%%%%%%%%%%% MODUL 2 %%%%%%%%%%%%%%%%%%%%%%%%%%%%%%

\section{Modul 2}
\begin{frame}{Modul 2}

\begin{itemize}
\item Hvad er kravene til det virtuelle håndtag?
\begin{itemize}
%Skuffen skal kunne åbnes og lukkes uden at brugeren berører den
%Skuffen skal automatisk lukke og låse efter 30 sek., hvis skuffen ikke er aktiv
\item Registrere hvornår skuffen er i brug, og ikke i brug
\item \textbf{30 sek. krav}
%Skuffen skal være i stand til kun at reagere efter tiltænkt anvendelse
\item Undgå uønsket aktivering
\item \textbf{Forbi passerende}
\end{itemize}

\item Der er valgt en IR-dioden som LED
\begin{itemize}
\item Udsender den infrarøde stråle
\item \textbf{Har en meget smal spredning}
\item \textbf{LED = 880$\pm$20 nm. Øje = 400 - 700 nm}
\end{itemize}

\item Der er valgt en phototransistoren som modtager
\begin{itemize}
\item Modtager den infrarøde stråle
\item \textbf{Har en meget smal spredning}
\item \textbf{Har daylight filter}
\end{itemize}

\item Moduldesign
\begin{itemize}
\item Der er valgt en operationsforstærker er valgt som komperator
\item \textbf{Komperatoren gør programmeringen lettere}
\end{itemize}

\end{itemize}
\end{frame}

\end{document}